\documentclass[sigconf]{acmart}

\usepackage{graphicx}
\usepackage{hyperref}
\usepackage{todonotes}

\usepackage{endfloat}
\renewcommand{\efloatseparator}{\mbox{}} % no new page between figures

\usepackage{booktabs} % For formal tables

\settopmatter{printacmref=false} % Removes citation information below abstract
\renewcommand\footnotetextcopyrightpermission[1]{} % removes footnote with conference information in first column
\pagestyle{plain} % removes running headers

\newcommand{\TODO}[1]{\todo[inline]{#1}}

\begin{document}
\title{Big Data Analytics in Developing Countries}


\author{Judy Phillips}
\orcid{1234-5678-9012}
\affiliation{%
  \institution{Indiana University}
  \streetaddress{P.O. Box 4822}
  \city{Bloomington} 
  \state{Indiana} 
  \postcode{47408}
}
\email{judkphil@iu.edu}









\begin{abstract}
Developing nations cope with many humanitarian challenges. Infrastructures are often inadequate to deal with basic public health, public safety and environmental concerns. As a result, citizens deal issues such as poverty, food insecurity, clean water access, and availability of health care. Impoverished populations are also vulnerable to corrupt government practices. The use of wireless and Internet related technology is growing globally. Mobile phone and social media usage in becoming common in even remote areas.  As a result, Big Data analytics is starting to have an impact in some of these countries. Long term, Big Data Analytics has the potential to play a significant role in enabling solutions that will mitigate the impact of some of these humanitarian issues. 
\end{abstract}

\keywords{i523,HID332, Big Data developing countries}


\maketitle



\section{Introduction}

Individuals in developing nations a long list of humanitarian challenges including poverty, hunger, health care access, and availability to a clean water sources. Other challenges include political instability, political corruption, and educational challenges.
Almost 1.3 billion people living in developing countries live on less than $1.50 a day /cite {top5}.  According to the UN approximately twenty two thousand children die each day in these countries due to poverty /cite{top10}.  More than eight hundred seventy million people of the entire third population have no food to eat or a very precarious food supply. A third of all childhood death in sub-Saharan Africa is caused by hunger related diseases. That is approximately two million six hundred thousand deaths per year. One child dies every five seconds \cite (top10}.  More than two hundred million children under five years of age in developing countries do not reach their developmental potential due to malnutrition \cite{WikiDevC}.  Over 1.2 billion people around the globe do not have regular access to clean drinking water. Many people die from common curable diseases that such as malaria, pneumonia, and diarrhea because they do not have access to health care. Approximately ten million children die each year from treatable diseases. \cite{top5}.  Fifty percent of pregnant women in developing countries lack proper maternal care. This results in over three hundred thousand maternal deaths annually from childbirth.  \cite{top10}. The threat of HIV is also reaching a pandemic level in many of the third world countries \cite(top5}.  

Put here an introduction about your topic. 
We just need one sample refernce so the paper compiles in LaTeX so we
put it here \cite{editor00}.

\section{figures}

In Figure \ref{f:fly} we show a fly. Please note that because we use
just columwidth that the size of the figure will change to the
columnwidth of the paper once we change the layout to final. CHnaging
the layout to final should not be done by you. All figures will be
listed at the end.

\begin{figure}[!ht]
  \centering\includegraphics[width=\columnwidth]{images/fly.pdf}
  \caption{Example caption}\label{f:fly}
\end{figure}

When copying the example, please do not check in the images from the
examples into your images directory as you will not need them for your
paper. Instead use images that you like to include. If you do not have
any images, do not dreate the images folder.

\section{Tables}

In case you need to create tables, you can do this with online tools
(if you do not mind sharing your data) such as
\url{https://www.tablesgenerator.com/} or other such tools (please
google for them). They even allow you to manage tables as CSV.

or generate them by hand while using the provided template in Table\ref{t:mytable}. Not ethat
the caption is before the tabular environment.

\begin{table}[htb]
\centering
\caption{My caption}
\label{t:mytabble}
\begin{tabular}{lll}
1 & 2 & 3 \\
\hline
4 & 5 & 6 \\
7 & 8 & 9
\end{tabular}
\end{table}

\section{Long example}

If you like to see a more elaborate example, please look at
report-long.tex. 

\section{Conclusion}

Put here an conclusion. Conlcusions and abstracts must not have any
citations in the section.


\begin{acks}

  The author would like to thank Dr. Gregor von Laszewski and the teaching assistants in the Data Science department at Indiana  University for their support and suggestions to write this paper.

\end{acks}

\bibliographystyle{ACM-Reference-Format}
\bibliography{report} 
******************************************************************************************************
\documentclass[sigconf]{acmart}



\begin{document}
\title{Big Data Analytics in Developing Countries}


\author{Judy Phillips}
\orcid{xxxx-xxxx-xxxx}
\affiliation{%
  \institution{Indiana University}
  \streetaddress{PO BOX 4822}
  \city{Bloomington} 
  \state{Indiana} 
  \postcode{47408}
}
\email{judkphil@iu.edu}

% The default list of authors is too long for headers}
\renewcommand{\shortauthors}{B. Trovato et al.}


\begin{abstract}
Developing nations cope with numerous humanitarian challenges. Infrastructures are often inadequate to deal with basic public health, public safety, and environmental concerns. As a result, citizens deal with issues such as poverty, food insecurity, and the availability of basic health care. Impoverished populations are also vulnerable to corrupt government practices. The use of wireless and Internet related technology is growing globally. Mobile phone and social media usage is becoming common even in remote areas. Big Data analytics is playing a role in mitigating the impacts of some of these humanitarian concerns. 
\end{abstract}

\keywords{I523, HID332, developing countries, food insecurity, public safety,  big data}


\maketitle

\section{INTRODUCTION}

In recent years there has been a huge increase in both the availability of digital technology in developing nations. According to the GSM Association seventy nine percent of the world’s total inhabited areas had mobile network coverage in 2012 \cite{DevEcon}. Social media such as facebook and twitter is being utilized worldwide. In addition, sensor technology is becoming less expensive and more efficient. Better algorthryms are being developed to utilize the low cost sensors for developmental activities. As a result, increasing amounts of data is being generated in developing countries. Data sources include call logs, mobile banking transactions, online user generated content such as blog posts and tweets \cite{GloPls}.

All of this data can be combined with data collected via traditional data sources such as datasets and surveys in such a way to provide information and insights that has never before been available. The diffusion of data science into the realm of international development constitutes an opportunity to bring powerful new tools to the fight against poverty, hunger, and disease \cite{GloPls}. Furthermore, the real time availability of much this data is available enables more timely and agile implementations of solutions. The combination of having more data that is delivered in a more timely real-time fashion are better outcomes.
 

\section{INFRASTRUCTURE}

Mobile phones have been diffused worldwide. It is estimated that ninety two percent of the worlds population owned a mobile phone in 2015 \cite{DevEcon}. Mobile phones are the cornerstone of my Big Data projects in the developing world because mobile phones are often the only interactive technology that low income individuals have access to.  

\section{BENEFITS}


\subsection{Public Health}



 

\subsection {Personal Health Care}



\subsection{Agriculture}


\subsection{Public Safety}

\subsection{Environmental Protection}

\subsection{Government}

\section{CHALLENGES AND ISSUES} 





\section{CONCLUSION}






\begin{acks}

  The author would like to thank Dr. Gregor von Laszewski and the teaching assistants in the Data Science department at Indiana  University for their support and suggestions to write this paper.

\end{acks}



\bibliographystyle{ACM-Reference-Format}
\bibliography{report.bib} 

\end{document}

