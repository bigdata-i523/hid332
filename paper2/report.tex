\documentclass[sigconf]{acmart}

\usepackage{graphicx}
\usepackage{hyperref}
\usepackage{todonotes}

\usepackage{endfloat}
\renewcommand{\efloatseparator}{\mbox{}} % no new page between figures

\usepackage{booktabs} % For formal tables

\settopmatter{printacmref=false} % Removes citation information below abstract
\renewcommand\footnotetextcopyrightpermission[1]{} % removes footnote with conference information in first column
\pagestyle{plain} % removes running headers

\newcommand{\TODO}[1]{\todo[inline]{#1}}

\begin{document}
\title{Big Data Analytics in Developing Countries}


\author{Judy Phillips}
\orcid{1234-5678-9012}
\affiliation{%
  \institution{Indiana University}
  \streetaddress{P.O. Box 4822}
  \city{Bloomington} 
  \state{Indiana} 
  \postcode{47408}
}
\email{judkphil@iu.edu}









\begin{abstract}
Developing nations cope with many humanitarian challenges. Infrastructures are often inadequate to deal with basic public health, public safety and environmental concerns. As a result, citizens deal issues such as poverty, food insecurity, clean water access, and availability of health care. Impoverished populations are also vulnerable to corrupt government practices. The use of wireless and Internet related technology is growing globally. Mobile phone and social media usage in becoming common in even remote areas.  As a result, Big Data analytics is starting to have an impact in some of these countries. Long term, Big Data Analytics has the potential to play a significant role in enabling solutions that will mitigate the impact of some of these humanitarian issues. 
\end{abstract}

\keywords{i523,HID332, Big Data developing countries}


\maketitle



\section{Introduction}

Individuals in developing nations a long list of humanitarian challenges including poverty, hunger, health care access, and availability to a clean water sources. Other challenges include political instability, political corruption, and educational challenges.
Almost 1.3 billion people living in developing countries live on less than $1.50 a day /cite {top5}.  According to the UN approximately twenty two thousand children die each day in these countries due to poverty /cite{top10}.  More than eight hundred seventy million people of the entire third population have no food to eat or a very precarious food supply. A third of all childhood death in sub-Saharan Africa is caused by hunger related diseases. That is approximately two million six hundred thousand deaths per year. One child dies every five seconds \cite (top10}.  More than two hundred million children under five years of age in developing countries do not reach their developmental potential due to malnutrition \cite{WikiDevC}.  Over 1.2 billion people around the globe do not have regular access to clean drinking water. Many people die from common curable diseases that such as malaria, pneumonia, and diarrhea because they do not have access to health care. Approximately ten million children die each year from treatable diseases. \cite{top5}.  Fifty percent of pregnant women in developing countries lack proper maternal care. This results in over three hundred thousand maternal deaths annually from childbirth.  \cite{top10}. The threat of HIV is also reaching a pandemic level in many of the third world countries \cite(top5}.  
Big Data Analytics is starting to be used to address some of these issues. In recent years, there has been a huge increase in both the availability of digital technology globally including in developing nations. According to the GSM Association 79 percent of the world’s total inhabited areas had mobile network coverage in 2012 \cite{DevEcon}.  Social media such as Facebook and twitter is being utilized by more and more people worldwide. Sensor technology is becoming less expensive and more efficient. Better algorthryms are being developed to utilize the low cost sensors for developmental activities. Data information sources include call logs, mobile banking transactions, blog posts, tweets, and Facebook content \cite{GloPls}.
Big Data analytics is starting to be used to address some of these humanitarian challenges to alleviate some of the suffering. All of this data newly available data is being combined with data collected via traditional data sources such as datasets and surveys. This is providing information and insights that has never before been available. The diffusion of data science into the realm of international development constitutes an opportunity to bring powerful new tools to the fight against poverty, hunger, and disease \cite{GloPls}. Furthermore, the real time availability of much this data is  enables more timely and agile implementations of solutions. This all results in significantly better outcomes.


Put here an introduction about your topic. 
We just need one sample refernce so the paper compiles in LaTeX so we
put it here \cite{editor00}.

\section{figures}

In Figure \ref{f:fly} we show a fly. Please note that because we use
just columwidth that the size of the figure will change to the
columnwidth of the paper once we change the layout to final. CHnaging
the layout to final should not be done by you. All figures will be
listed at the end.

\begin{figure}[!ht]
  \centering\includegraphics[width=\columnwidth]{images/fly.pdf}
  \caption{Example caption}\label{f:fly}
\end{figure}

When copying the example, please do not check in the images from the
examples into your images directory as you will not need them for your
paper. Instead use images that you like to include. If you do not have
any images, do not dreate the images folder.

\section{Tables}

In case you need to create tables, you can do this with online tools
(if you do not mind sharing your data) such as
\url{https://www.tablesgenerator.com/} or other such tools (please
google for them). They even allow you to manage tables as CSV.

or generate them by hand while using the provided template in Table\ref{t:mytable}. Not ethat
the caption is before the tabular environment.

\begin{table}[htb]
\centering
\caption{My caption}
\label{t:mytabble}
\begin{tabular}{lll}
1 & 2 & 3 \\
\hline
4 & 5 & 6 \\
7 & 8 & 9
\end{tabular}
\end{table}

\section{HEALTHCARE}

Big Data has enormous potential to address health care challenges in the developing world [DevEcon]. One of the primary problems with healthcare in the developing world is the overall lack of access. This is caused by a combination of geographical accessability and the lack of basic medical resources. There are shortages trained medical professionals, medical equipment,  and drug stocks. People in rural areas often have to travel long distances in order to obtain care. There are also lack of resources to implement basic public health regimes such as immunization policies.  All of this makes the occurrence of serious disease outbreaks and epidemics common and difficult to manage when the do occur. Another issue is the existence of widespread fake drug distribution networks. 
\subsection{Public Health}
One area in which Big Data can have an enormous impact on the health of vulnerable populations is in public health policy. Proper public health infrastructure is also needed to prevent, treat, and manage serious disease outbreaks. Public health policies and related public education can also influence attitudes and behaviors concerning important health related matters such as maternal health and immunizations. .  
Big Data is extremely useful for managing serious disease outbreaks including pandemics and epidemics. Big Data and data science can be used first to track and monitor the spread of the disease and then to effectively allocate resources and medication so that the disease can be properly treated and contained. If fact, the term for the Infodemiology. It is a whole new field of data science.  

.  
Health related data that is mined from social media and sites such as twitter and then combined with data visualization techniques to track the geographic spread of a disease. As the spread of the disease is being tracked in real time, big data is used to ensure that all available resources are allocated effectively. Big data ensures the right distribution of resources including medical personal and medication at the right time to the right location. Proper resource allocation is especially important when lifesaving medical supplies are in short supply. According to the US Center for Disease Control and prevention (CDC), online data can help detect disease outbreaks up to two weeks before confirmed diagnosis or lab confirmation \cite{Glopls}. When this resource allocation technique was used in Tanzania during a malaria outbreak it reduced the drug facilities that were out of stock with the appropriate medication from 78% to 26% \cite{DevEcon}.  
Social Media can also be used to track peoples health related beliefs, perceptions and concerns at any a given time and in real time. This methodology is referred to as sentiment analysis. For example. Researchers can get an indication of health related attitudes about immunizations, the use of medication or prenatal care programs.  Big Data can also be used to measure the impacts of humanitarian aid and intervention. For example, the United Nations used this technique to evaluate whether the Every Woman Every Child initiative had had an impact. This was a program that was designed to increase awareness of maternal health, breastfeeding, vaccinations. A team of researchers analyzed social media posts for two years for relevant keywords such as “breastfeeding” or “vaccination” to determine if the program has resulted in increased parental awareness \cite{DevEcon}. The information collected can be used to identify needs and to establish and manage public health policies and programs.
Sentiment analysis can also be used to track other public health related issues such housing shortages, employment, and inflated food prices.  This methodology has proved to identify issues earlier than traditional methods and thus enables more timely deployment of resources and solutions. \cite {GloPls). 
\subsection{Health Care Access}
There are often problems with geographical accessabiltiy to health care. People in rural areas often need to travel long distances to obtain to visit a health care professional. Individuals in rural areas do not have enough health care providers. Specialists are rarely available. The Internet of Things technology can solve some of these issues with items such as personal sensors and patient monitoring via patient sensors. 

\section{ENVIRONMENTAL PROTECTION AND WATER SUPPLY}

Almost a billion people in the world to not have a reliable source of clean drinking water. According to World Water Development Report in 2012,  inadequate sanitation and poor hygiene result in  3.5 million deaths annually. Much of the water is wasted or leaked due to faulty pipes. Other water is lost due to unidentified or unnecessary pollutants. Big Data can be used to help identify and monitor such issues so that remedies can be implemented.  
In one example IBM worked with the city of Tshwane in South Africa to develop a crowd source application that users used to report water supply issues such as faulty pipes. The result was the discovery of thirty million dollars of wasted water sources. This application is operated without the need of a central inspection authority /cite{HuffPst}.
The Internet of Things can used for the purpose of monitoring water supply and quality. Sensors are frequently used to monitor pollutants in a river or water source. Resources are deployed to remedy problems when they are detected.  One example is in Vietnam’s Da Nang which is a major port city on the South China Sea. The Da Nang water company uses Big Data to provider real time analysis of the city’s water supply. The goal is to better manage leaks accurately forecast future demand. Big Data sensors are installed throughout each stage of the water treatment process. Water quality is tracked in real time and they receive notifications when anything changes  /cite{DevEcon}.

\section{AGRICULTURE}

More than half the population in all of the developing nations depend upon agriculture and farming for at least two meals a day. This accounts for almost seventy five percent of the worlds poorest people /cite{top10}. Therefore one important way to address poverty and food insecurity is to find ways to make farming techniques more effective and productive. Big Data has big potential to dramatically increase production for small scale farmers. 
Studies suggest that ineffective farm operations such a late planting, lack of proper land preparation, improper harvesting techniques and poor housing and feeding of livestock can reduce a smallholders farmers productivity by up to forty percent. 
One technique for doing this is Precision agriculture. The objective of Precision agriculture is to provide farmers with informed personalized information so that they can make better operating decisions in real time. Data is collected on things such as soil conditions, weather, seeding rates, and crop yields using technology such as sensors, drones and satellites. Sensors can be located in fields, inside livestock, or on farm equipment.  After the data is collected it is analyzed and returned to the farmers via computers and mobile phones in terms of customized and personalized solutions. Instructions may be such things as the optimum type of seeds, pesticides, herbicides, and fertilizer use. The objective is to match to inputs with the exact need. Resources are used efficiently production is maximized. Another solution involves collecting data to locate and notify farmers of the spread of crop and livestock plight. The objective is that farmers to take safety measures as soon as possible (hftpost).
In Uganda there is a Big Data tools project that uses Precision agriculture techniques that was developed by Grameen Foundation. Data is collected on farmers, farming practices, and external conditions. It is given back to farmers in the form of a community knowledge database via Android phones. Information about time and methods of planting crops, caring for farm animals and marketing their products /cite{Hfpst}.
Another way in which big data can be used for small holder farmers to support financing opportunities. In Nairbia Africa a company Gro Ventures is building a platform in which integrates information of crops and environmental information to give lenders confidence to lend money to farmers. One of the offerings allows farmers to pool their data to apply for collective loans to buy shared tractors and equipment /cite{HfPst}.   


\section{Conclusion}

Put here an conclusion. Conlcusions and abstracts must not have any
citations in the section.


\begin{acks}

  The author would like to thank Dr. Gregor von Laszewski and the teaching assistants in the Data Science department at Indiana  University for their support and suggestions to write this paper.

\end{acks}

\bibliographystyle{ACM-Reference-Format}
\bibliography{report} 
******************************************************************************************************
\documentclass[sigconf]{acmart}



\begin{document}
\title{Big Data Analytics in Developing Countries}


\author{Judy Phillips}
\orcid{xxxx-xxxx-xxxx}
\affiliation{%
  \institution{Indiana University}
  \streetaddress{PO BOX 4822}
  \city{Bloomington} 
  \state{Indiana} 
  \postcode{47408}
}
\email{judkphil@iu.edu}

% The default list of authors is too long for headers}
\renewcommand{\shortauthors}{B. Trovato et al.}


\begin{abstract}
Developing nations cope with numerous humanitarian challenges. Infrastructures are often inadequate to deal with basic public health, public safety, and environmental concerns. As a result, citizens deal with issues such as poverty, food insecurity, and the availability of basic health care. Impoverished populations are also vulnerable to corrupt government practices. The use of wireless and Internet related technology is growing globally. Mobile phone and social media usage is becoming common even in remote areas. Big Data analytics is playing a role in mitigating the impacts of some of these humanitarian concerns. 
\end{abstract}

\keywords{I523, HID332, developing countries, food insecurity, public safety,  big data}


\maketitle

\section{INTRODUCTION}

In recent years there has been a huge increase in both the availability of digital technology in developing nations. According to the GSM Association seventy nine percent of the world’s total inhabited areas had mobile network coverage in 2012 \cite{DevEcon}. Social media such as facebook and twitter is being utilized worldwide. In addition, sensor technology is becoming less expensive and more efficient. Better algorthryms are being developed to utilize the low cost sensors for developmental activities. As a result, increasing amounts of data is being generated in developing countries. Data sources include call logs, mobile banking transactions, online user generated content such as blog posts and tweets \cite{GloPls}.

All of this data can be combined with data collected via traditional data sources such as datasets and surveys in such a way to provide information and insights that has never before been available. The diffusion of data science into the realm of international development constitutes an opportunity to bring powerful new tools to the fight against poverty, hunger, and disease \cite{GloPls}. Furthermore, the real time availability of much this data is available enables more timely and agile implementations of solutions. The combination of having more data that is delivered in a more timely real-time fashion are better outcomes.
 

\section{INFRASTRUCTURE}

Mobile phones have been diffused worldwide. It is estimated that ninety two percent of the worlds population owned a mobile phone in 2015 \cite{DevEcon}. Mobile phones are the cornerstone of my Big Data projects in the developing world because mobile phones are often the only interactive technology that low income individuals have access to.  

\section{BENEFITS}


\subsection{Public Health}



 

\subsection {Personal Health Care}



\subsection{Agriculture}


\subsection{Public Safety}

\subsection{Environmental Protection}

\subsection{Government}

\section{CHALLENGES AND ISSUES} 





\section{CONCLUSION}






\begin{acks}

  The author would like to thank Dr. Gregor von Laszewski and the teaching assistants in the Data Science department at Indiana  University for their support and suggestions to write this paper.

\end{acks}



\bibliographystyle{ACM-Reference-Format}
\bibliography{report.bib} 

\end{document}

