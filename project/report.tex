\documentclass[sigconf]{acmart}

\usepackage{graphicx}
\usepackage{hyperref}
\usepackage{todonotes}

\usepackage{endfloat}
\renewcommand{\efloatseparator}{\mbox{}} % no new page between figures

\usepackage{booktabs} % For formal tables

\settopmatter{printacmref=false} % Removes citation information below abstract
\renewcommand\footnotetextcopyrightpermission[1]{} % removes footnote with conference information in first column
\pagestyle{plain} % removes running headers

\newcommand{\TODO}[1]{\todo[inline]{#1}}

\begin{document}
\title{Big Data Analytics in Developing Countries}


\author{Judy Phillips}
\orcid{xxxx-xxxx-xxxx}
\affiliation{%
  \institution{Indiana University}
  \streetaddress{PO BOX 4822}
  \city{Bloomington} 
  \state{Indiana} 
  \postcode{47408}
}
\email{judkphil@iu.edu}


\begin{abstract}
In the United States more money is spent on Healthcare than in any other industrialized country in the world. Yet, health care access is often problematic and health care quality indicators are lower or mediocre as compared to other countries with similar economic status. Insights offered by Big Data Analytics can find solutions that will significantly lower costs and improve delivery of health care in the United States.  These solutions have to potential can save billions of dollars in health care costs and to improve the quality of care for millions of Americans. 
\end{abstract}

\keywords{I523, HID332, health care costs, predictive analytics, electronic health records,  big data}


\maketitle

\section{Introduction}

The United States spends substantially more on health care than any other nation in the world.  In 2013 the United States spent 2.9 trillion dollars on health care.  According to the Organization for Economic Cooperation and Development (OEDC), the United States spends 2.5 times per person than the average of the thirty five OEDC related nations.  In 2016 the United States spent 9822 dollars per person annually on health care. The average amount spent per person among all OEDC nations was 4033 dollars.  The next highest spender was Switzerland at 7919 dollars per person. This was almost 2000 dollars less per person.  Health care spending accounts for 17.6 percent of the United States Gross National Product (GDP). United States Health care spending continues to grow and has outpaced overall GDP growth rates for several decades /cite{McDonald}. Among OEDC nations the average spending as a percentage of Gross National Product was 9 percent. The next highest was Switzerland at 12 percent.    
Despite the excessive spending, the United States ranks among the worst on measures of health care quality, health access equity, quality of life /cite{McDonald}.  Average life expectancy in the United States is 76.3 years.  The average life expectancy among all OEDC countries is 77.9 years. The incidence of obstetric trauma is 9.6 per 10000 births in the United States compared to 5.7 incidents per 100000 in other countries. The statistics for preventable hospital admissions also compare poorly in comparison to other nations. In the United States the hospital admission rate for Asthma and COPD was 262 per 100000 in comparison to the average of 236 per 100000.  There United States has fewer healthcare providers and healthcare access in problematic.  In the United States there are 2.6 practicing physicians and 2.8 hospital beds per 1000 population. This compares to an average of 3.4 physicians and 4.7 hospital beds on the average in the other countries. Most other OEDC countries have achieved almost universal insurance coverages. On the average, 98 percent of persons in OEDC countries have health insurance. In the United States only 90 percent have health insurance. This means that 10 percent of the United States population does not have health insurance coverage. In addition, cost sharing makes access additionally prohibitive.  In 2016, 22.3 percent of the persons in the United States had skipped a medical consultation due to cost concerns. In comparison, the average percentage of individuals who had skipped medical visits due to cost in OEDC nations was 10.5 percent.  In the United States 11.6 percent of the population had skipped prescribed medications due to cost in 2016. This compares to an average of 7.1 percent of the population in other countries who had foregone a prescribed medication because of cost. Thirty eight percent of the population in the United States is obese. The average obesity rate in other countries at nineteen percent \cite{OEDC}.
Big data has the potential to help manage some of these extraordinary health care costs while at the same time improving patient health outcomes. McKinsey estimates that big data has the potential to save the United States between 348 and 493 billion dollars annually in Health Care costs \cite{CIO}. Big data can be used to:  reduce waste and overutilization, improve health care coordination, better match treatments to patient needs, expose fraud and abuse, avoid hospital admissions, and to enhance chronic disease management protocols. 


\section{Infrastruture}


\section{Big Data}
There are unprecedented amounts health related information data available. The amount of healthcare data is expected to grow to 25000 petabytes by 2020 /cite{rock}. Sources of data include hospitals, labs, research companies, insurance companies, and government agencies. Pharmaceutical companies maintain research and development information in medical databases. Detailed information patient information is available in public insurance program databases. Health care providers maintain electronic health related data available.  Examples of unstructured data include. Pharmaceutical companies maintain research and development information in medical databases. The United States government houses databases concerning clinical drug trials. Detailed information patient information is available in public insurance program databases. Health care providers maintain electronic health related data available. Data is collected by the United States Centers Disease Control and Prevention /cite{CIO}. It is estimated that approximately eighty percent of the data is in an unstructured format /cite{McDonald}.  Unstructured data comes from medical devices, doctors, notes, imagining reports, and medical correspondence /cite{McDonald}, sensor devices, and sentiment analysis from social media.  Much of this information has been underutilized or not being used to its full potential until now. For example, big data gives us the ability to capitalize and make use of the valuable clinical information that is available in physician and nurses notes /cite{HlthCat}.  Big Data gives us the ability to combine and analyze this wide variety of data from many multiple sources to provide new and invaluable insights. Information can be used to more accurate and timely diagnosis, predicting and identifying patients at risk, and to better match and administer treatment plans /cite{McDonald}. 
Health care professionals are now able to capture and analyze mountains of digitized health care data. Recent advances in Big Data technology gives us the ability to capture, share and store healthcare data at an unprecedented pace.

\section{Healthcare}

"Big Data has enormous potential to address health care challenges in the developing world" \cite{DevEcon}. One of the primary problems with healthcare in the developing world is the overall lack of access. This is caused by a combination of geographical accessibility and the lack of basic medical resources. There are shortages trained medical professionals, medical equipment, and drug stocks. People in rural areas often have to travel long distances in order to obtain care. There are also a lack of resources to implement basic public health regimes such as immunization policies.  All of this makes the occurrence of serious disease outbreaks and epidemics common and difficult to manage when they do occur.  Another issue is the existence of widespread fake drug distribution networks.

\subsection{Public Health}
One area in which Big Data can have an enormous impact on the health of vulnerable populations is in public health policy. Proper public health infrastructure is needed to prevent, treat, and manage serious disease outbreaks. Public health policies and related public education can also educate populations and influence attitudes and behaviors concerning important health related matters such as maternal health and immunizations. 

Big Data is extremely useful for managing serious disease outbreaks, including pandemics and epidemics. Big Data and data science can be used first to track and monitor the spread of the disease and then to effectively allocate resources and medication so that the disease can be properly treated and contained. In fact, the term for this field is Infodemiology. It is a whole new field of data science \cite{www-google-GloPls}.

Health related data is mined from social media and sites such as twitter and then combined with data visualization techniques to track the geographic spread of a disease. As the spread of the disease is being tracked in real time, big data is used to ensure that all available resources are allocated effectively. Big data ensures the right distribution of resources, including medical personnel and medication at the right time to the right location \cite{www-google-Dataflo}. Proper resource allocation is especially important when lifesaving medical supplies are in short supply. According to the US Center for Disease Control and prevention (CDC), online data can help detect disease outbreaks before confirmed diagnosis or lab confirmation \cite{www-google-GloPls}. It is estimated that disease outbreaks can be identified up to two weeks sooner than with the use of traditional methods such as physician reporting \cite{DevEcon}.  When this resource allocation technique was used in Tanzania during a malaria outbreak, it reduced the number of drug facilities that were out of stock of the appropriate medication during the epidemic from 78 percent to 26 percent \cite{DevEcon}.

Social Media can also be used to track peoples health related beliefs, perceptions and concerns at any a given time and in real time. This methodology is referred to as sentiment analysis. For example, researchers can get an indication of health related attitudes about immunizations, the use of medication or prenatal care programs by reviewing social media posts.  These studies can assist with health related education efforts. Social media and big data analytics are also be used to measure the impacts of humanitarian aid and intervention. For example, the United Nations used this technique to evaluate whether the Every Woman Every Child initiative had had an impact. This was a program that was designed to increase awareness of maternal health, breastfeeding, vaccinations. A team of researchers analyzed social media posts for two years for relevant keywords, such as breastfeeding or vaccination to determine if the program has resulted in increased parental awareness \cite{DevEcon}. The information collected can be used to identify needs in order to establish and manage public health policies and programs.

Sentiment analysis can also be used to track other public health related issues such housing shortages, employment, and inflated food prices. This methodology is able to identify issues earlier than traditional methods and thus enables more timely deployment of resources and solutions \cite{www-google-GloPls}. 

\subsection{Health Care Access}
In developing countries, there are often problems with geographical accessibility to health care. People in rural areas often need to travel long distances to visit a health care professional. Also, rural areas do not have enough primary health care providers and specialists are rarely available. 

The Internet of Things technology can solve some of these issues. One solution is patient sensors. Relatively low cost sensors can be worn on the person to monitor physiological variables in real time.  The data collected can be transmitted to health care providers in a distant locations for diagnosis and treatment. These sensors can be used for routine as well as critical health issues such as heart palpitations. For example, in Africa there is a device called Cardio pad. It is a medical tablet that can be used to perform and collect information from cardiology related tests by individuals who have no cardiac training. The information gathered can then be sent to a cardiac specialist via mobile phone in order to receive diagnosis and treatment instructions\cite{DevEcon}. In China the Internet of Things technology Institute is developing a telephone booth sized health capsule. Rural villagers can be receive a diagnosis from a distantly located physician when they step into it. \cite{DevEcon}. 

\subsection{Distribution of Fake Drugs}
The widespread distribution of fake drugs is a huge health hazard in developing nations. According to the World Health Organization, counterfeit antimalarial and tuberculosis drugs account for seven hundred thousand deaths annually. Big Data technology is playing a huge role in fighting this crime.  One nonprofit organization has developed a possible solution. The name of the program is called GoldKeys. All legitimate prescription containers have a twelve digit scratch off code. Customers can verify the authenticity of the medication by texting the scratched off code number to a health hotline.  The number is matched to information in a cloud database and the information is sent back to the customer. The project is being maintained and funded primarily by Hewlett Packard \cite{DevEcon}. 

\section{Environmental Protection and Water Supply}

Almost a billion people in the world to not have a reliable source of clean drinking water \cite{www-google-top10}. According to World Water Development Report in 2012, inadequate sanitation and poor hygiene result in 3.5 million deaths annually \cite{DevEcon}. Much of the water is wasted or leaked due to faulty pipes. Other water is lost due to unidentified or unnecessary pollutants.

The Internet of Things can be used for the purpose of monitoring water supply and quality. Sensors are frequently used to monitor pollutants in a river or water source. Resources are deployed to remedy problems when they are detected.  One example is in the city of Da Nang, Vietnam. Da Nang is a major port city on the South China Sea. The Da Nang water company uses Big Data to provide real time analysis of the citys water supply. The goal is to better manage leaks, monitor pollutants, and accurately forecast future demand. Big Data sensors are installed throughout each stage of the water treatment process. Water quality is tracked in real time. Notifications are sent if there are problems. \cite{DevEcon}.

In another example, IBM worked with the city of Tshwane in South Africa to develop a crowd source application that users use to report water supply issues such as faulty pipes. The result was the discovery of thirty million dollars of wasted water sources. This application operates without the need of a central inspection authority \cite{www-google-Hffpst}.

\section{Public Safety and Crisis Intervention}

One of the most important areas in which Big Data is being deployed is to enhance public safety and crisis intervention efforts during natural disasters. "The availability of digital data collected and analyzed rapidly and in real time can drastically improve interventions and outcomes in crisis situations for vulnerable populations" \cite{www-google-GloPls}.  

One of the most widely used tools in this effort are crisis maps. Crisis maps use data from numerous sources, including local citizen reports, social network data, and environmental data to aid emergency responders in times of natural disaster. "Crisis maps have been deployed during dozens of events worldwide,including the 2012 Haiti earthquake and the 2010 Pakistan floods" \cite{www-google-Hffpst}.
In Haiti during an earthquake a centralized text message center was set up that allowed cell phone users to report where people were trapped. The United States Geological Survey has developed a system the monitors twitter for spikes about earthquakes globally. This information can be used to evaluate the location, quantify magnitude, identify epicenter, and respond quickly and appropriately \cite{www-google-GloPls}.

\section{Agriculture}

"More than half the population in all of the developing nations depend upon agriculture and farming for at least two meals a day. This accounts for almost seventy five percent of the worlds poorest people" \cite{www-google-top10}.  Therefore, one important way to address poverty and food insecurity is to find ways to make farming techniques more effective and productive. Big Data has big potential to dramatically increase production for small scale farmers.

"Studies suggest that ineffective farm operations such a late planting, lack of proper land preparation, improper harvesting techniques and poor housing and feeding of livestock can reduce a smallholders farmers productivity by up to forty percent" \cite{DevEcon}.
One technique for improving production is Precision agriculture. The objective of Precision agriculture is to provide farmers with informed, personalized information so that they can make better operational decisions in real time. Data is collected on things such as soil conditions, weather, seeding rates, and crop yields using technology such as sensors, drones and satellites\cite{DevEcon}. Sensors can be located in fields, inside livestock, or on farm equipment.  After the data is collected it is analyzed and returned to the farmers via computers and mobile phones in terms of customized solutions. Instructions may be such things as the optimum type of seeds, pesticides, herbicides, and fertilizer use. The objective is to match inputs with the exact need. When resources are used efficiently production is maximized. Another solution involves collecting data to locate and notify farmers of the spread of crop and livestock plight. The objective is that farmer take safety measures as soon as possible \cite{www-google-Hffpst}.

In Uganda there is a Big Data tools project that uses Precision agriculture techniques that were developed by the Grameen Foundation. Data is collected on farmers, farming practices, and external conditions. It is given back to farmers in the form of a community knowledge database via Android phones. Information about the time and methods of planting crops, caring for farm animals and marketing their products \cite{DevEcon}.

Another way in which big data can be used for small holder farmers to support financing opportunities. In Nairobi, Africa the company Gro Ventures is building a platform which integrates information about crops and the environmental conditions to give lenders more confidence to lend money to farmers. One of the offerings allows farmers to pool their data to apply for collective loans to buy shared tractors and equipment \cite{www-google-Hffpst}.  

\section{Challenges}
There are many challenges to the successful implementation of many of these projects.  Many people in the least developed nations still lack access to internet service or a mobile phone. There are high costs associated with using big data technology. Cost of mobile phones, analytical services and data services often cost prohibitive for individual citizens. There is also a Big Data skill set deficient. Big data technology and the analytics to turn big data into actionable information requires technical skills that are often not available. Furthermore, health care professionals and other related personnel often lack knowledge or training about data science. 

In order for initiatives to be successful, financial and technical support will need to come from other sources: academia, public and private sector, and philanthropic. To date, there are numerous non-goverment organizations (NGOs) working throughout the world to fight poverty and reduce disease \cite{www-google-Dataflo}. The United Nations started an initiative in 2009 called Global Pulse. The objective of Global pulse is to research ways that Big Data can be incorporated into the developing world to improve lives. They are currently conducting several research initiatives in various locations throughout the world. Several private organizations are also playing a role. For example, Google has announced a plan to develop high speed internet solutions in developing countries using high altitude balloons. Their goal is to add an additional 1 billion people to the Internet from Africa, and Southwest Asia \cite{DevEcon}. 

\section{Conclusion}

Although Big Data does not have the ability to solve all of the worlds problems, it does have enormous potential to reduce suffering and save lives for those living in developing countries. Big data is giving smallholder farmers resources to substantially increase their food production. This will play a substantial role in the fight against poverty and food insecurity. Big data analytics is improving health by making health care accessible to even those in the most remote locations. Big data provides the knowledge to identify and monitor water availability issues such as waste and pollution so that problems can be identified dealt with immediately. Big is also saving lives by providing the real time knowledge needed to respond effectively to health epidemics and natural disasters. As the use of internet related devices continues to increase throughout the developing world, the impact of big data will continue to grow.   







\begin{acks}

  The author would like to thank Dr. Gregor von Laszewski and the teaching assistants in the Data Science department at Indiana University for their support and suggestions to write this paper.

\end{acks}



\bibliographystyle{ACM-Reference-Format}
\bibliography{report.bib} 

\end{document}

