\documentclass[sigconf]{acmart}

\usepackage{graphicx}
\usepackage{hyperref}
\usepackage{todonotes}

\usepackage{endfloat}
\renewcommand{\efloatseparator}{\mbox{}} % no new page between figures

\usepackage{booktabs} % For formal tables

\settopmatter{printacmref=false} % Removes citation information below abstract
\renewcommand\footnotetextcopyrightpermission[1]{} % removes footnote with conference information in first column
\pagestyle{plain} % removes running headers

\newcommand{\TODO}[1]{\todo[inline]{#1}}

\begin{document}
\title{Big Data Analytics in Developing Countries}


\author{Judy Phillips}
\orcid{xxxx-xxxx-xxxx}
\affiliation{%
  \institution{Indiana University}
  \streetaddress{PO BOX 4822}
  \city{Bloomington} 
  \state{Indiana} 
  \postcode{47408}
}
\email{judkphil@iu.edu}


\begin{abstract}
In the United States more money is spent on Healthcare than in any other industrialized country in the world. Yet, health care access is often problematic and health care quality indicators are lower or mediocre as compared to other countries with similar economic status. Insights offered by Big Data Analytics can find solutions that will significantly lower costs and improve delivery of health care in the United States.  These solutions have to potential can save billions of dollars in health care costs and to improve the quality of care for millions of Americans. 
\end{abstract}

\keywords{I523, HID332, health care costs, predictive analytics, electronic health records,  big data}


\maketitle

\section{Introduction}

The United States spends substantially more on health care than any other nation in the world.  In 2013 the United States spent 2.9 trillion dollars on health care.  According to the Organization for Economic Cooperation and Development (OEDC), the United States spends 2.5 times per person than the average of the thirty five OEDC related nations.  In 2016 the United States spent 9822 dollars per person annually on health care. The average amount spent per person among all OEDC nations was 4033 dollars.  The next highest spender was Switzerland at 7919 dollars per person. This was almost 2000 dollars less per person.  Health care spending accounts for 17.6 percent of the United States Gross National Product (GDP). United States Health care spending continues to grow and has outpaced overall GDP growth rates for several decades /cite{McDonald}. Among OEDC nations the average spending as a percentage of Gross National Product was 9 percent. The next highest was Switzerland at 12 percent.    
Despite the excessive spending, the United States ranks among the worst on measures of health care quality, health access equity, quality of life /cite{McDonald}.  Average life expectancy in the United States is 76.3 years.  The average life expectancy among all OEDC countries is 77.9 years. The incidence of obstetric trauma is 9.6 per 10000 births in the United States compared to 5.7 incidents per 100000 in other countries. The statistics for preventable hospital admissions also compare poorly in comparison to other nations. In the United States the hospital admission rate for Asthma and COPD was 262 per 100000 in comparison to the average of 236 per 100000.  There United States has fewer healthcare providers and healthcare access in problematic.  In the United States there are 2.6 practicing physicians and 2.8 hospital beds per 1000 population. This compares to an average of 3.4 physicians and 4.7 hospital beds on the average in the other countries. Most other OEDC countries have achieved almost universal insurance coverages. On the average, 98 percent of persons in OEDC countries have health insurance. In the United States only 90 percent have health insurance. This means that 10 percent of the United States population does not have health insurance coverage. In addition, cost sharing makes access additionally prohibitive.  In 2016, 22.3 percent of the persons in the United States had skipped a medical consultation due to cost concerns. In comparison, the average percentage of individuals who had skipped medical visits due to cost in OEDC nations was 10.5 percent.  In the United States 11.6 percent of the population had skipped prescribed medications due to cost in 2016. This compares to an average of 7.1 percent of the population in other countries who had foregone a prescribed medication because of cost. Thirty eight percent of the population in the United States is obese. The average obesity rate in other countries at nineteen percent \cite{OEDC}.
Big data has the potential to help manage some of these extraordinary health care costs while at the same time improving patient health outcomes. McKinsey estimates that big data has the potential to save the United States between 348 and 493 billion dollars annually in Health Care costs \cite{CIO}. Big data can be used to:  reduce waste and overutilization, improve health care coordination, better match treatments to patient needs, expose fraud and abuse, avoid hospital admissions, and to enhance chronic disease management protocols. 


\section{Infrastruture}


\section{Big Data}
There are unprecedented amounts health related information data available. The amount of healthcare data is expected to grow to 25000 petabytes by 2020 /cite{rock}. Sources of data include hospitals, labs, research companies, insurance companies, and government agencies. Pharmaceutical companies maintain research and development information in medical databases. Detailed information patient information is available in public insurance program databases. Health care providers maintain electronic health related data available.  Examples of unstructured data include. Pharmaceutical companies maintain research and development information in medical databases. The United States government houses databases concerning clinical drug trials. Detailed information patient information is available in public insurance program databases. Health care providers maintain electronic health related data available. Data is collected by the United States Centers Disease Control and Prevention /cite{CIO}. It is estimated that approximately eighty percent of the data is in an unstructured format /cite{McDonald}.  Unstructured data comes from medical devices, doctors, notes, imagining reports, and medical correspondence /cite{McDonald}, sensor devices, and sentiment analysis from social media.  Much of this information has been underutilized or not being used to its full potential until now. For example, big data gives us the ability to capitalize and make use of the valuable clinical information that is available in physician and nurses notes /cite{HlthCat}.  Big Data gives us the ability to combine and analyze this wide variety of data from many multiple sources to provide new and invaluable insights. Information can be used to more accurate and timely diagnosis, predicting and identifying patients at risk, and to better match and administer treatment plans /cite{McDonald}. 
Health care professionals are now able to capture and analyze mountains of digitized health care data. Recent advances in Big Data technology gives us the ability to capture, share and store healthcare data at an unprecedented pace.

\section{Health Cost Drivers}

Why is health care so much more expensive in the United States than it is anywhere else in the world?  One of the main drivers is the unorganized and uncoordinated payment structure.  United States citizens are covered under a variety of different payment systems. Most individuals have health insurance through the private insurance market.  Private insurance coverage is usually obtained thru an individuals employer. Individual can purchase private insurance directly.   Disabled citizens and citizens over the age of 65 receive health insurance the through a federal government program called Medicare. Low income individuals and children can obtain health insurance through Medicaid. Medicaid is administered at the state level and thus rules and regulations vary by state.  Many individuals are not insured at all because do not qualify for any of these programs and purchasing it directly a private insurance carrier is extremely cost prohibitive.  All of these payment systems have different rules and regulations, methodologies and price structures by which to pay health care providers.
The system is inefficient and flawed because the basic economic concepts such as supply and demand and competition do not work in this sector. This is because none of the players are incentivized to manage or reduce costs /cite{Milbank}.  Consumers do not manage medical utilization because it is being paid for by a third party, the insurance company.  Insurance coverage that covers most of the costs insulate patients from the true costs of medical care /cite{Milbank}. Providers are not incentivized to provide efficient, cost effective care.  Traditional fee for service provider payment methodologies that reward health caregivers for quantity instead of quality often result in overutilization of unnecessary tests and treatment procedures.  Insurance companies pass the cost of services on to the consumers in the form of higher premiums. The cost inflation cycle goes on and on.  
Medical errors cost the system 10.5 billion dollars a year. This amount could be as much as 1 trillion dollars a year if lost productivity is taken into account /cite{pracfus}.
Fraud

\section{Value Based Reimbursement}
Currently, the most common provider payment structure is fee for service. Under this concept, providers are paid a fee for each and every service that they perform. This tends to encourage overutilization instead of the efficient use of medical resources. The United States tends to perform more and more expensive diagnostic services and treatment services than any other country in the in the world. The United States is well known for over testing and over treatment /cite{PBSO}. The physicians use their own personal judgement to determine what tests and treatment services to order.  From a clinical perspective, many of these tests are not medically necessary. This is a wasteful use of resources. 
For this reason many payors including private health insurance companies, Medicare and Medicaid are starting to base reimbursement of value based incentives.  Value based reimbursement systems reimburse health care providers for health care quality instead of health care quantity. Providers are awarded for high quality and positive patient outcomes instead of service volume \cite{McDonald}.  Payors are also are starting to reward pharmaceutical companies by basing reimbursements of drug effectiveness \cite{McKinsey}. 
There are several different types Value based models. One form is the patient centered medical home. In this model the primary care physician coordinates the patients care and is rewarded for performance for improving quality and reducing costs for individual patients. Another value based system is a population model that include population health initiatives and reward providers for successfully improving the health of the entire population \cite{value}. An example of this type of program is an Accountable Care Organization (ACO). In Accountable Care Organizations groups of doctors, hospitals, and other providers work together to provide coordinated care for Medicare patients. Providers share in Medicare savings with they deliver high quality care and manage costs wisely /cite{Medicare}.
The goals of value based care include improved efficiencies, improved care coordination, elimination of  unnecessary tests and duplication of services.  All of this results in lower costs, better health outcomes, and enhanced patient satisfaction. 
Big Data Analytics is integral to making this process work. Value based health care depends upon quality data collection and precise data analytics /cite{value}.  First, the data must be collected and analyzed in order to define what defines quality care. Big Data is collected and analyzed in order to establishing clinical guidelines that promote a more rational use of specific diagnostic tests and treatment protocols.  Second, this information must be made available to health care givers in a format that they can use for day to day clinical decision making. This is often in the form of a cloud based integration platform \cite{value}.  Next, data must be collected on an ongoing basis to provide feedback indicating whether  the providers are meeting the defined standards and if not, what can be done to improve performance. In addition, the same data can benefit future patients when data analytics are taken beyond the individual patient and are used to develop care protocols for entire patient populations \cite{value}. 


\subsection{Fraud and Abuse}
The National Healthcare Anti-Fraud Association estimates losses due to health care fraud at 80 billion dollars annually. Other industry sources estimate fraudulent related losses to be around 200 billion. This accounts for approximately 2 to 3 percent of total health care spending. Research indicates that only 5 percent of these losses are ever recovered /cite(datameer}.
Common types of abuse include: billing for services that are not rendered, billing for more expensive procedures than were actually delivered, and the performance of unnecessary services. 
In the past the identification of misrepresented claims was tedious and time consuming work.  According to an article by RevCycle Intelligence, ‘’fraud involves a misrepresentation of some key fact or event and when repeated misrepresentations are made, they create patterns that can be detected when compared to legitimate claims’’ /cite{datameer}.  Big Data analytics makes it possible to easily identify and tag such claims. Anthem Health Insurance company, one of the nations biggest players uses big data and machine learning algorithms to tag suspicious claims on a daily basis as the claims are processed. Tagged claims are then sent to clinical coding experts for review. The objective is to identify and weed out fraudulent claims before they are actually paid \cite{datameer}.   
The Center for Medicare and Medicaid Services used predictive data analytics to identify and recover 210.7 million in health care fraud. They did this by assigning risk scores to claims, providers via algorithms, identify abnormal billing patterns and claim submissions.  United Healthcare realized a 2200 percent return on their investment in a Hadoop Big Data platform that was used to identify and tag inaccurate claims using a systemic and repeatable methodology \cite{McDonald}.
Other uses of Big Data analytics include identifying links between providers to access whether an identified unethical activity is being practiced by related providers.  Big Data analytics can utilize machine learning algorithms combined with historical information to detect anomalies and suspicious data patterns. Big data analytics can be used to identify a hospital’s overutilization of services in a short time period, patients who are receiving health care services from different hospitals in different locations at the same time, or identical prescriptions filled for the same patient in multiple locations.
 

\subsection{Internet Connected Medical Devices}
Internet connected medical devices are becoming more affordable and are being used more and more commonly.  Gartner, the analysis firm estimates that there will be more than 25 billion connected health devices by the year 2020 \cite{HlthCat}. These devices collect data in real time and send information into the cloud. Devices include blood pressure monitors, pulse oximeters, glucose monitors, electronic scales. \cite{HlthCat}.  Some of these devices are being used as personal monitors. For example, smartphones can track user exercise.  Patients are able to take a more active role in their own health and wellbeing.  Other devices being can be worn by patients with health conditions at home and in medical facilities to stream data continuously to provide real time patient monitoring.  Processing or real time events can be supplemented with machine learning algorithms to help provide physicians with information they need to make lifesaving interventions \cite{McDonald}.  The devices can improve care by enabling patients with the ability to self manage their conditions. Patient care tends to be more proactive as patient vital signs are can be monitored constantly. Medical alerts can be sent to care providers such that they immediately aware of changes in a patients condition and can respond accordingly. One example of the use of personal devices in patient care are pediatricians monitoring asthmatics to identify environmental triggers for attacks \cite{CIO}.
\section{Environmental Protection and Water Supply}

Almost a billion people in the world to not have a reliable source of clean drinking water \cite{www-google-top10}. According to World Water Development Report in 2012, inadequate sanitation and poor hygiene result in 3.5 million deaths annually \cite{DevEcon}. Much of the water is wasted or leaked due to faulty pipes. Other water is lost due to unidentified or unnecessary pollutants.

The Internet of Things can be used for the purpose of monitoring water supply and quality. Sensors are frequently used to monitor pollutants in a river or water source. Resources are deployed to remedy problems when they are detected.  One example is in the city of Da Nang, Vietnam. Da Nang is a major port city on the South China Sea. The Da Nang water company uses Big Data to provide real time analysis of the citys water supply. The goal is to better manage leaks, monitor pollutants, and accurately forecast future demand. Big Data sensors are installed throughout each stage of the water treatment process. Water quality is tracked in real time. Notifications are sent if there are problems. \cite{DevEcon}.

In another example, IBM worked with the city of Tshwane in South Africa to develop a crowd source application that users use to report water supply issues such as faulty pipes. The result was the discovery of thirty million dollars of wasted water sources. This application operates without the need of a central inspection authority \cite{www-google-Hffpst}.

\section{Public Safety and Crisis Intervention}

One of the most important areas in which Big Data is being deployed is to enhance public safety and crisis intervention efforts during natural disasters. "The availability of digital data collected and analyzed rapidly and in real time can drastically improve interventions and outcomes in crisis situations for vulnerable populations" \cite{www-google-GloPls}.  

One of the most widely used tools in this effort are crisis maps. Crisis maps use data from numerous sources, including local citizen reports, social network data, and environmental data to aid emergency responders in times of natural disaster. "Crisis maps have been deployed during dozens of events worldwide,including the 2012 Haiti earthquake and the 2010 Pakistan floods" \cite{www-google-Hffpst}.
In Haiti during an earthquake a centralized text message center was set up that allowed cell phone users to report where people were trapped. The United States Geological Survey has developed a system the monitors twitter for spikes about earthquakes globally. This information can be used to evaluate the location, quantify magnitude, identify epicenter, and respond quickly and appropriately \cite{www-google-GloPls}.

\section{Agriculture}

"More than half the population in all of the developing nations depend upon agriculture and farming for at least two meals a day. This accounts for almost seventy five percent of the worlds poorest people" \cite{www-google-top10}.  Therefore, one important way to address poverty and food insecurity is to find ways to make farming techniques more effective and productive. Big Data has big potential to dramatically increase production for small scale farmers.

"Studies suggest that ineffective farm operations such a late planting, lack of proper land preparation, improper harvesting techniques and poor housing and feeding of livestock can reduce a smallholders farmers productivity by up to forty percent" \cite{DevEcon}.
One technique for improving production is Precision agriculture. The objective of Precision agriculture is to provide farmers with informed, personalized information so that they can make better operational decisions in real time. Data is collected on things such as soil conditions, weather, seeding rates, and crop yields using technology such as sensors, drones and satellites\cite{DevEcon}. Sensors can be located in fields, inside livestock, or on farm equipment.  After the data is collected it is analyzed and returned to the farmers via computers and mobile phones in terms of customized solutions. Instructions may be such things as the optimum type of seeds, pesticides, herbicides, and fertilizer use. The objective is to match inputs with the exact need. When resources are used efficiently production is maximized. Another solution involves collecting data to locate and notify farmers of the spread of crop and livestock plight. The objective is that farmer take safety measures as soon as possible \cite{www-google-Hffpst}.

In Uganda there is a Big Data tools project that uses Precision agriculture techniques that were developed by the Grameen Foundation. Data is collected on farmers, farming practices, and external conditions. It is given back to farmers in the form of a community knowledge database via Android phones. Information about the time and methods of planting crops, caring for farm animals and marketing their products \cite{DevEcon}.

Another way in which big data can be used for small holder farmers to support financing opportunities. In Nairobi, Africa the company Gro Ventures is building a platform which integrates information about crops and the environmental conditions to give lenders more confidence to lend money to farmers. One of the offerings allows farmers to pool their data to apply for collective loans to buy shared tractors and equipment \cite{www-google-Hffpst}.  

\section{Challenges}
There are many challenges to the successful implementation of many of these projects.  Many people in the least developed nations still lack access to internet service or a mobile phone. There are high costs associated with using big data technology. Cost of mobile phones, analytical services and data services often cost prohibitive for individual citizens. There is also a Big Data skill set deficient. Big data technology and the analytics to turn big data into actionable information requires technical skills that are often not available. Furthermore, health care professionals and other related personnel often lack knowledge or training about data science. 

In order for initiatives to be successful, financial and technical support will need to come from other sources: academia, public and private sector, and philanthropic. To date, there are numerous non-goverment organizations (NGOs) working throughout the world to fight poverty and reduce disease \cite{www-google-Dataflo}. The United Nations started an initiative in 2009 called Global Pulse. The objective of Global pulse is to research ways that Big Data can be incorporated into the developing world to improve lives. They are currently conducting several research initiatives in various locations throughout the world. Several private organizations are also playing a role. For example, Google has announced a plan to develop high speed internet solutions in developing countries using high altitude balloons. Their goal is to add an additional 1 billion people to the Internet from Africa, and Southwest Asia \cite{DevEcon}. 

\section{Conclusion}

Although Big Data does not have the ability to solve all of the worlds problems, it does have enormous potential to reduce suffering and save lives for those living in developing countries. Big data is giving smallholder farmers resources to substantially increase their food production. This will play a substantial role in the fight against poverty and food insecurity. Big data analytics is improving health by making health care accessible to even those in the most remote locations. Big data provides the knowledge to identify and monitor water availability issues such as waste and pollution so that problems can be identified dealt with immediately. Big is also saving lives by providing the real time knowledge needed to respond effectively to health epidemics and natural disasters. As the use of internet related devices continues to increase throughout the developing world, the impact of big data will continue to grow.   







\begin{acks}

  The author would like to thank Dr. Gregor von Laszewski and the teaching assistants in the Data Science department at Indiana University for their support and suggestions to write this paper.

\end{acks}



\bibliographystyle{ACM-Reference-Format}
\bibliography{report.bib} 

\end{document}

