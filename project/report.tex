\documentclass[sigconf]{acmart}



\begin{document}
\title{Big Data Analytics Role in Reducing Healthcare Costs in the United States}


\author{Judy Phillips}
\orcid{xxxx-xxxx-xxxx}
\affiliation{%
  \institution{Indiana University}
  \streetaddress{PO BOX 4822}
  \city{Bloomington} 
  \state{Indiana} 
  \postcode{47408}
}
\email{judkphil@iu.edu}


\begin{abstract}
In the United States more money is spent on health care than in any other industrialized country in the world. Yet, health care access is often problematic and health care quality indicators are lower or mediocre as compared to other countries with similar economic status. Insights offered by Big Data Analytics can find solutions that will significantly lower costs and improve delivery of health care in the United States.  These solutions have to potential can save billions of dollars in health care costs and to improve the quality of care for millions of Americans. 
\end{abstract}

\keywords{I523, HID332, health care costs, predictive analytics, electronic health records,  big data}


\maketitle

\section{Introduction}

Big data has the potential to help manage health care costs while at the same time improving patient health outcomes. Big data can be used to support and improve hospital functions such as clinical decision support, disease surveillance, and population health management \cite{www-google-springer}. Big data has the potential to save resources by reducing healthcare waste and overutilization, improving coordination efforts among caregivers, better matching treatments to patient needs, exposing fraud and abuse, avoiding hospital admissions.  Big Data gives us the ability to combine and analyze a wide variety of data from many multiple sources in a way that has never before been possible. This new information can provide new and invaluable insights. Information can be used to achieve more accurate and timely diagnosis, predict and identify at risk patients, and to better match and administer treatment plans /cite{www-google-McDonald}. The result will be dramatic cost savings. McKinsey and Company estimates that big data has the potential to save the United States between 348 and 493 billion dollars annually in Health Care costs \cite{www-google-CIO}.

\section{Health Care Statistics}

The United States spends substantially more on health care than any other nation in the world.  In 2013 the United States spent 2.9 trillion dollars on health care.  According to the Organization for Economic Cooperation and Development (OEDC), the United States spends 2.5 times per person than the average of the thirty five OEDC related nations.  In 2016 the United States spent 9822 dollars per person annually on health care. The average amount spent per person among all OEDC nations was 4033 dollars.  The next highest spender was Switzerland at 7919 dollars per person. This was almost 2000 dollars less per person.  Health care spending accounts for 17.6 percent of the United States Gross National Product (GDP). United States Health care spending continues to grow and has outpaced overall GDP growth rates for several decades \cite{www-google-McDonald}. Among OEDC nations the average spending as a percentage of Gross National Product was 9 percent. The next highest was Switzerland at 12 percent. 

Despite the excessive spending, the United States ranks among the worst on measures of health care quality, health access equity, quality of life \cite{www-google-McDonald}.  Average life expectancy in the United States is 76.3 years.  The average life expectancy among all OEDC countries is 77.9 years. The incidence of obstetric trauma is 9.6 per 10000 births in the United States compared to 5.7 incidents per 100000 in other countries. The statistics for preventable hospital admissions also compare poorly in comparison to other nations. In the United States the hospital admission rate for Asthma and COPD was 262 per 100000 in comparison to the average of 236 per 100000.  There United States has fewer healthcare providers and healthcare access in problematic.  In the United States there are 2.6 practicing physicians and 2.8 hospital beds per 1000 population. This compares to an average of 3.4 physicians and 4.7 hospital beds on the average in the other countries. Most other OEDC countries have achieved almost universal insurance coverages. On the average, 98 percent of persons in OEDC countries have health insurance. In the United States only 90 percent have health insurance. This means that 10 percent of the United States population does not have health insurance coverage. In addition, cost sharing makes access additionally prohibitive.  In 2016, 22.3 percent of the persons in the United States had skipped a medical consultation due to cost concerns. In comparison, the average percentage of individuals who had skipped medical visits due to cost in OEDC nations was 10.5 percent.  In the United States 11.6 percent of the population had skipped prescribed medications due to cost in 2016. This compares to an average of 7.1 percent of the population in other countries who had foregone a prescribed medication because of cost. Thirty eight percent of the population in the United States is obese. The average obesity rate in other countries at nineteen percent \cite{www-google-OEDC}.

Big data has the potential to help manage some of these extraordinary health care costs while at the same time improving patient health outcomes. McKinsey estimates that big data has the potential to save the United States between 348 and 493 billion dollars annually in Health Care costs \cite{www-google-CIO}. Big data can be used to:  reduce waste and overutilization, improve health care coordination, better match treatments to patient needs, expose fraud and abuse, avoid hospital admissions, and to enhance chronic disease management protocols. 



\section{Big Data}

Big data refers to electronic data sets that are so large and complex that they cannot be managed with traditional hardware and software. A report delivered to the United States Congress in August 2012 defines big data a “large volumes of high velocity, complex, variable data they require advanced techniques and technologies to enable capture, storage, distribution, management, and analysis of the information”. Big data characteristics include variety, velocity, and veracity, and volume.  Health care data is involves the processing of overwhelmingly large complex data sets, from a wide variety of sources and a very rapid speed \cite{www-google-springer}.  Health care professionals are now able to capture and analyze these mountains of digitized health care data. Recent advances in Big Data technology gives us the ability to capture, share and store healthcare data at an unprecedented pace. 

\subsection{Volume}
The health care industry has always generated large amounts of data.  Data is needed for record keeping, compliance and regulatory reporting and patient care. Historically this data has been stored in hard copy format. Now more and more data is being created and stored digitally. Estimates are that in 2011 there were 150 exabytes of data. The amount of big data being generated and is growing rapidly. It is expected to soon reach the zettabyte scale and then soon after that into the yottabytes \cite{www-google-springer}.}

\subsection{Velocity}
Traditionally, health care data has been static: paper files, x-ray films, prescriptions. Ironically, in many medical situations, the speed of the response can mean the difference between life and death. Increasingly, more and more of the data is being collected in real time and at a rapid pace. For example medical monitoring devices information is being collected continuously and many time requires immediate response \cite{www-google-springer}. 

\subsection{Variety}

There is an enormous variety of data being collected. The data is in multimedia formats. The data may be includes structured, unstructured, or semi-structured. Examples of data include web and social media. These may include facebook, twitter, health plan websites and smart phone applications. There is machine to machine data from patient sensors. Biometric data is available such as fingerprints, genetics, hand writing information and imagining reports \cite{www-google-springer}. Sources of data include physicians, hospitals, laboratories, research companies, insurance companies, and government agencies. Physicians generate electronic medical records, physician notes and medical correspondence. Pharmaceutical companies maintain research and development information in medical databases. Detailed information patient information is available in public insurance and hospital databases. The United States government houses databases concerning clinical drug trials. Data is collected by the United States Centers Disease Control and Prevention \cite{www-google-CIO}. Data is being continuously generated by various medical sensor devices. 

\subsection{Veracity}

The characteristic Veracity addresses whether the information is credible and error free. Veracity is extremely important in healthcare because life or death decisions on being based upon the information provided. There is a particular concern because interpretations of unstructured data such as physician notes could be incorrect or inprecise. Big data architecture, platforms, methodologies and tools are designed to take into account the uncertainties of big data analytics \cite{www-google-springer}. 

\subsection{Infrastructure}

Much of this information has been underutilized or not being used to its full potential until now. It is estimated that approximately eighty percent of the data is in an unstructured format /cite{McDonald}. For example, big data gives us the ability to capitalize and make use of the valuable clinical information that is unstructured, such as the information that is available in physician and nurses notes \cite{www-google-HlthCat}.  Advances in Big Data technology including data management and cloud computing are facilitating the development of platforms for more effective capture, storage, and manipulation of large data sets from multiple sources \cite{www-google-springer}.  All of this is providing information and insights that has never before been possible to obtain.



\section{Health Cost Drivers}

Why is health care so much more expensive in the United States than it is anywhere else in the world?  One of the main drivers is the unorganized and uncoordinated payment structure.  United States citizens are covered under a variety of different payment systems. Most individuals have health insurance through the private insurance market.  Private insurance coverage is usually obtained thru an individuals employer. Individual can purchase private insurance directly.   Disabled citizens and citizens over the age of 65 receive health insurance the through a federal government program called Medicare. Low income individuals and children can obtain health insurance through Medicaid. Medicaid is administered at the state level and thus rules and regulations vary by state.  Many individuals are not insured at all because do not qualify for any of these programs and purchasing it directly a private insurance carrier is extremely cost prohibitive.  All of these payment systems have different rules and regulations, methodologies and price structures by which to pay health care providers.
The system is inefficient and flawed because the basic economic concepts such as supply and demand and competition do not work in this sector. This is because none of the players are incentivized to manage or reduce costs \cite{milbank}.  Consumers do not manage medical utilization because it is being paid for by a third party, the insurance company.  Insurance coverage that covers most of the costs insulate patients from the true costs of medical care \cite{milbank}. Providers are not incentivized to provide efficient, cost effective care.  Traditional fee for service provider payment methodologies that reward health caregivers for quantity instead of quality often result in overutilization of unnecessary tests and treatment procedures.  Insurance companies pass the cost of services on to the consumers in the form of higher premiums. The cost inflation cycle goes on and on.  
Medical errors cost the system 10.5 billion dollars a year. This amount could be as much as 1 trillion dollars a year if lost productivity is taken into account \cite{www-google-pracfuspracfus}.
Fraud

\section{Value Based Reimbursement}
Currently, the most common provider payment structure is fee for service. Under this concept, providers are paid a fee for each and every service that they perform. This tends to encourage overutilization instead of the efficient use of medical resources. The United States tends to perform more and more expensive diagnostic services and treatment services than any other country in the in the world. The United States is well known for over testing and over treatment \cite{www-google-PBSO}. The physicians use their own personal judgement to determine what tests and treatment services to order.  From a clinical perspective, many of these tests are not medically necessary. This is a wasteful use of resources. 
For this reason many payors including private health insurance companies, Medicare and Medicaid are starting to base reimbursement of value based incentives.  Value based reimbursement systems reimburse health care providers for health care quality instead of health care quantity. Providers are awarded for high quality and positive patient outcomes instead of service volume \cite{www-google-McDonald}.  Payors are also are starting to reward pharmaceutical companies by basing reimbursements of drug effectiveness \cite{www-google-McKinsey}. 
There are several different types Value based models. One form is the patient centered medical home. In this model the primary care physician coordinates the patients care and is rewarded for performance for improving quality and reducing costs for individual patients. Another value based system is a population model that include population health initiatives and reward providers for successfully improving the health of the entire population \cite{www-google-liason}. An example of this type of program is an Accountable Care Organization (ACO). In Accountable Care Organizations groups of doctors, hospitals, and other providers work together to provide coordinated care for Medicare patients. Providers share in Medicare savings with they deliver high quality care and manage costs wisely \cite{www-google-ACO}.
The goals of value based care include improved efficiencies, improved care coordination, elimination of  unnecessary tests and duplication of services.  All of this results in lower costs, better health outcomes, and enhanced patient satisfaction. 
Big Data Analytics is integral to making this process work. Value based health care depends upon quality data collection and precise data analytics \cite{www-google-liason}.  First, the data must be collected and analyzed in order to define what defines quality care. Big Data is collected and analyzed in order to establishing clinical guidelines that promote a more rational use of specific diagnostic tests and treatment protocols.  Second, this information must be made available to health care givers in a format that they can use for day to day clinical decision making. This is often in the form of a cloud based integration platform \cite{www-google-liason}.  Next, data must be collected on an ongoing basis to provide feedback indicating whether  the providers are meeting the defined standards and if not, what can be done to improve performance. In addition, the same data can benefit future patients when data analytics are taken beyond the individual patient and are used to develop care protocols for entire patient populations \cite{www-google-liason}. 


\section{Fraud and Abuse}
The National Healthcare Anti-Fraud Association estimates losses due to health care fraud at 80 billion dollars annually. Other industry sources estimate fraudulent related losses to be around 200 billion. This accounts for approximately 2 to 3 percent of total health care spending. Research indicates that only 5 percent of these losses are ever recovered \cite{www-google-datameer}.
Common types of abuse include: billing for services that are not rendered, billing for more expensive procedures than were actually delivered, and the performance of unnecessary services. 
In the past the identification of misrepresented claims was tedious and time consuming work.  According to an article by RevCycle Intelligence, ‘’fraud involves a misrepresentation of some key fact or event and when repeated misrepresentations are made, they create patterns that can be detected when compared to legitimate claims’’ \cite{www-google-datameer}.  Big Data analytics makes it possible to easily identify and tag such claims. Anthem Health Insurance company, one of the nations biggest players uses big data and machine learning algorithms to tag suspicious claims on a daily basis as the claims are processed. Tagged claims are then sent to clinical coding experts for review. The objective is to identify and weed out fraudulent claims before they are actually paid \cite{www-google-datameer}.   
The Center for Medicare and Medicaid Services used predictive data analytics to identify and recover 210.7 million in health care fraud. They did this by assigning risk scores to claims, providers via algorithms, identify abnormal billing patterns and claim submissions.  United Healthcare realized a 2200 percent return on their investment in a Hadoop Big Data platform that was used to identify and tag inaccurate claims using a systemic and repeatable methodology \cite{www-google-McDonald}.
Other uses of Big Data analytics include identifying links between providers to access whether an identified unethical activity is being practiced by related providers.  Big Data analytics can utilize machine learning algorithms combined with historical information to detect anomalies and suspicious data patterns. Big data analytics can be used to identify a hospital’s overutilization of services in a short time period, patients who are receiving health care services from different hospitals in different locations at the same time, or identical prescriptions filled for the same patient in multiple locations.
 

\section{Internet Connected Medical Devices}
Internet connected medical devices are becoming more affordable and are being used more and more commonly.  Gartner, the analysis firm estimates that there will be more than 25 billion connected health devices by the year 2020 \cite{www-google-HlthCat}. These devices collect data in real time and send information into the cloud. Devices include blood pressure monitors, pulse oximeters, glucose monitors, electronic scales. \cite{www-google-HlthCat}.  Some of these devices are being used as personal monitors. For example, smartphones can track user exercise.  Patients are able to take a more active role in their own health and wellbeing.  Other devices being can be worn by patients with health conditions at home and in medical facilities to stream data continuously to provide real time patient monitoring.  Processing or real time events can be supplemented with machine learning algorithms to help provide physicians with information they need to make lifesaving interventions \cite{www-google-McDonaldMcDonald}.  The devices can improve care by enabling patients with the ability to self manage their conditions. Patient care tends to be more proactive as patient vital signs are can be monitored constantly. Medical alerts can be sent to care providers such that they immediately aware of changes in a patients condition and can respond accordingly. One example of the use of personal devices in patient care are pediatricians monitoring asthmatics to identify environmental triggers for attacks \cite{CIO}.

\section{Electronic Health Records}

Between 2001 and 2014 Electronic Health record usage in physician offices rose from 20 percent to 82 percent. This amount of data that is being collected by large health systems and treatment centers around the country is massive \cite{www-google-pred}. 
Electronic Health records are digitized versions of what is available in a patients charts from multiple providers. The records are made available to health providers in real time. The objective is to provide in one place an electronic record of a patients health. An electronic health record contains medical history, diagnosis, medications, immunizations dates, allergy information, radiology images and test results.  An Electronic Health record system can also automatically check for problems such as medication conflicts and notify clinicians with alerts. Systems often include and electronic prescription subscribing systems. They can include and be integrated with evidence base tools that help providers make decisions about patients care \cite{www-google-elec}.  
The concept is simple, but it improves health care for the patient in many ways. Physicians have better organized, more accessible, and more complete information about the patient.  Improves ability to make an accurate diagnosis.  Easily accessible patient information reduces medical errors and unnecessary tests. It reduces the incidence of duplicate tests. It improves coordination of care because it insures that all caregivers are aware of the conditions that are being treated by other caregivers. It makes it easier to communicate critical clinical information to all applicable providers in a timely fashion. Because information is made available in real time to providers in drastically reduces the probability of errors caused by such things as allergic reactions and drug interactions especially in emergency situations.  Easier to track and manage when patients are due for vaccinations or screenings. Electronic subscribing allows physicians to communicate directly with pharmacies. Prescriptions are no longer lost or misread \cite{www-google-elec}.
Electronic medical records save money by reducing transcription costs, eliminating chart storage and access costs, reduced medical errors, and improved patient care. At the same time quality of care and patient safety is enhanced, and health outcomes are better \cite{www-google-elec}. 




\section{Predictive Analytics}

Predictive analytics is the process of learning from historical data in order to make predictions about the future. The objective of predictive health analytics is to provide insights that enable personalized medical care for each individual patient. Traditionally, physicians have always used predictive analytics as they provide health care based upon what they know about the medical history of each individual. Predictive Health analytics seeks to supplement that knowledge with software tools that enable physicians to make more informed choices about the patients treatment based upon data from population cohorts /cite{pred}.  The patients are directed to specific treatment plans based upon their specific conditions as compared to other patients in a similar cohort.  This additional knowledge can have the potential to provide the physician with the information to provide a more effective treatment plan \cite{www-google-pred}. Predictive analytics can also aid in the diagnosis of a patient. Data sources are disparate and include: clinical, claims, research and patient generated data. Sources of patient generated are sensors and social media.  
Predictive analytics has the potential to materially reduce health care costs and improve patient care.   Insights provided can in clinical decision support, prevent hospital readmission preventions, aid in adverse incidence avoidance, and help chronic disease management. 
  
  

\section{Challenges}
HIPAA compliance in regards to patients privacy and security and confidentiality. HIPAA regulations regarding patient confidentialltiy it is against the law can not be compromised. can not be compromised. Privacy and Security of Patient data is paramount. Big data technology has inconsistent security technology is problematic. There are some vendors that have experience with security /cite{HlthCat}. 
Another issue is lack of technical expertise. The manipulation and extraction of data from often unstructured data sets requires special expertise. There have been some recent changes in tooling that will make it easier for individualized with less specialized skills to manipulate the data. For example,  Big data is starting to use include SQL as a tool for querying and manipulating the data. Microsoft Polybase and Impala uses SQL Hadoop. 
 

\section{Conclusion}








\begin{acks}

  The author would like to thank Dr. Gregor von Laszewski and the teaching assistants in the Data Science department at Indiana University for their support and suggestions to write this paper.

\end{acks}



\bibliographystyle{ACM-Reference-Format}
\bibliography{report.bib} 

\end{document}

