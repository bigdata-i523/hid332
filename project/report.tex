\documentclass[sigconf]{acmart}



\begin{document}
\title{Big Data Analytics Role in Reducing Healthcare Costs in the United States}


\author{Judy Phillips}
\orcid{xxxx-xxxx-xxxx}
\affiliation{%
  \institution{Indiana University}
  \streetaddress{PO BOX 4822}
  \city{Bloomington} 
  \state{Indiana} 
  \postcode{47408}
}
\email{judkphil@iu.edu}


\begin{abstract}
In the United States more money is spent on health care than in any other industrialized country in the world. Yet, health care access is often problematic and health care quality indicators are lower or mediocre as compared to other countries with similar economic status. Insights offered by Big Data Analytics can find solutions that will significantly lower costs and improve delivery of health care in the United States.  These solutions have to potential can save billions of dollars in health care costs and to improve the quality of care for millions of Americans. 
\end{abstract}
\keywords{I523, HID332, health care costs, predictive analytics, electronic health records,  big data}
\maketitle

\section{Introduction}
Health care spending in the United States greatly exceeds the spending of other industrialized countries.  Americans spend 3 trillion dollars annually on health care. Health expenditures currently account for 17.6 percent of the Gross National Product (GDP) and are expected to increase at an average rate of 5.8 percent through 2025. Health care spending has exceeded growth of the Gross National Product (GDP) in 42 of the previous 50 years \cite{www-google-transparent}.  Health spending threatens the nations fiscal health \cite{springer}.  Despite the excessive spending, the United States ranks among the worst on measures of health care quality, health access equity, and quality of life \cite{www-google-McDonald}.  Policy makers do not know how to respond.

Big data analytics has the potential to help manage address some of the cost issues while simultaneously improving patient health outcomes.  Big Data ability gives us the ability to combine and analyze data from a wide variety of sources in ways that have never before been possible. This new information is providing new and valuable insights into ways to provide more effective and efficient patient care. The associations, patterns, and trends in big data may hold the key to reducing expenditures, improving care, and saving lives \cite{springer}.  The information is being used to achieve more accurate and timely diagnoses, better match treatment plans to patient needs, and predict and identify at risk patients and populations \cite{www-google-McDonald}.  Mobile applications are being used to monitor patient care in real time.   Big data can reduce health care waste, improve coordination of care, expose fraud and abuse, and to speed up the research and development pipeline.

The cost savings estimates are substantial. McKinsey and Company estimates that Big data analytics has the potential to reduce health care costs in the United State by 12 to 17 percent. This equates to a savings of between 348 to 493 billion dollars annually \cite(www-google-CIO}. 

Some of the tools and methodologies that big data uses to introduce efficiencies into the American health care system include:  Outcome based reimbursement methodologies, electronic health records, medical device monitoring, predictive analytics, evidence based medicine, genomic analysis, and claim prepayment fraud analysis.  Big data technologies is adding value and improving efficiency in almost every area of health care including, clinical decision support, administration, pharmaceutical research and development, and population health management. 
 


\section{Comparison to Other Countries}
According to the Organization for Economic Cooperation and Development (OEDC), the United States spends 2.5 times per person than the average of OEDC related industrialized nations.  In 2016, the United States spent 9822 dollars per person annually on health care.  In comparison, the average amount spent per person among all OEDC nations was 4033 dollars.  The next highest spender was Switzerland at 7919 dollars per person \cite{OEDC}. The average spending as a percentage of Gross National Product (GDP) among OEDC nations was 9 percent. Switzerland was again the next highest spender at 12 percent of their Gross National Product (GDP) being spent on health care.  According to a McKinsey and Company analysis, the United States spends 600 billion dollars more annually than the estimated benchmark amount as calculated based upon the countries size and wealth as compared to other OEDC related nations \cite{google-McKinsey}.

The United States lags in many standard indicators of health quality.  According to a Commonwealth Fund study of 11 developed countries in 2013, the United States ranked fifth in quality and worst in infant mortality. We also ranked last in the prevention of deaths from treatable conditions such as strokes, diabetes, high blood pressure and treatable cancers.  The average life expectancy in the United States is 76.3 years.  The average life expectancy among all OEDC countries is 77.9 years.  The incidence of obstetric trauma is 9.6 per 10000 births in the United States compared to 5.7 incidents per 100000 in other countries. The statistics for preventable hospital admissions also compare poorly in comparison to other nations. In the United States the hospital admission rate for Asthma and COPD was 262 per 100000 in comparison to the average of 236 per 100000. Thirty eight percent of the population in the United States is obese. The average obesity rate in other countries at nineteen percent.  The United States has fewer physicians and hospitals.  In the United States, there are 2.6 practicing physicians and 2.8 hospital beds per 1000 population. This compares to an average of 3.4 physicians and 4.7 hospital beds on the average in the other countries \cite{OEDC}. 

The United States has material problems with health care access.  Most other OEDC countries have achieved almost universal insurance coverages. On the average, 98 percent of persons in OEDC countries have health insurance. In the United States only 90 percent have health insurance. This means that 10 percent of the United States population does not have health insurance coverage. In addition, cost sharing requirements often make access additionally prohibitive.   In 2016, 22.3 percent of the persons in the United States had skipped a medical consultation due to cost concerns. In comparison, the average percentage of individuals who had skipped medical visits due to cost in OEDC nations was 10.5 percent.  In the United States 11.6 percent of the population had skipped taking a prescribed medication due to cost in 2016. This compares to an average of 7.1 percent of the population in other OEDC countries who reported foregoing foregone a prescribed medication due to cost \cite{OEDC}.


\section{Health Cost Drivers}
Why is health care so much more expensive in the United States than it is anywhere else in the world?  Some of the contributing factors include: the basic health care economic payment structure, the inefficient and wasteful use of resources, medical errors, lack of transparency within the system, and unnecessary administrative costs, and fraud and abuse.

\subsection{Health Care Payment Structure}
Many of the cost issues can be contributed to the complex, uncoordinated, multi-payer payment structure. Private insurance companies, Medicaid, and Medicare are the primary payers.  An individuals eligibility by payer is dependent upon factors such as employment status, income level, age, and whether or not they are disabled.  Most citizens obtain private insurance through their employment. Individuals who are 65 years of age or older or disabled are eligible for Medicare. These individuals may also purchase private Medicare Supplement insurance on their own to pay expenses that Medicare does not cover. Low income individuals may be eligible for Medicaid. If an individual is not eligible for any of these programs, he can purchase individual health insurance from a private insurance company on his own.  However, individual health insurance is expensive.  According data from E-care, in 2016, the average monthly premium for an individual was 393 dollars per month. The average cost for family coverage was and 1021 dollars per month \cite{www-google-cost}. In addition, individual insurance policies often include fairly high cost sharing features. Even though subsidies are available through the Affordable Care Act to offset some of these costs, many people choose to forego insurance entirely due to the prohibitive expense.


The system is inefficient and flawed because the basic economic concepts such as supply and demand and competition do not work in this sector. This is because none of the players are incentivized to manage or reduce costs \cite{milbank}.  Consumers do not manage medical utilization because it is being paid for by a third party, the insurance company.  Insurance coverage thus insulates patients from the true costs of medical care \cite{milbank}. Providers are not incentivized to provide efficient, cost effective care. Most providers on paid via a traditional fee for service methodology. That is providers, are paid for each service that they provide. Traditional fee for service provider payment methodologies that reward health caregivers for quantity instead of quality often result in overutilization of unnecessary tests and treatment procedures.  The structure is such that it encourages the production of inefficient and low value services \cite{milbank}.   Insurance companies pass the cost of services on to the consumers in the form of higher premiums year after year. The cost inflation cycle goes on and on.  


Administrative waste is another result of the complexity of the United States multi payer payment structure. Each payer has their own rules and standards. Benefit and coverage options can vary dramatically among individuals even within the same insurance company.  According the OEDC 2008 estimates, the United States spends 7.3 percent of health care expenses on administrative activities. This is more than any other country. Comparatively, Germany spends 5.6 percent, Canada spends 2.6 percent and France spends 1.9 percent \cite{OEDC}.   Administrative activities include transaction related activities such as billing and claims payment, and regulatory compliance such as those required to comply with government and nongovernment accreditation and regulation including licensing requirements.

\subsection{Waste and Inefficiencies}
McKinsey and Company estimates that we can Big Data has the potential to save 273 dollars annually by reducing clinical and research and development waste \cite{springer}.There are three types of waste: administrative, operational, and clinical. 
Administrative waste often results from the complexity of the United States multi payer insurance and payment structure.  Administrative activities include transaction related activities such as billing and claims payment, and regulatory compliance such as those required to comply with government and nongovernment accreditation and regulation including licensing requirements, Health Insurance Portability and Accountability Act (HIPPA) compliance, Occupational Safety and Health Administration (OSHA),  Joint Commission on Accreditation of Hospitals (JCHCO), and National Committee for Quality Assurance (NCQA).  According the OEDC 2008 estimates, the United States spends 7.3 percent of health care expenses on administrative activities. This is more than any other country. Comparatively, Germany spends 5.6 percent, Canada spends 2.6% and France spends 1.9%. 
 Operation waste results from duplication of services or inefficient production processes.  An example would be a duplicate medical service because of lost medical records or the same service already being provided by another caregiver.
Clinical waste is created by the creation of low value outputs. Clinical waste is the spending on goods and services that provide marginal or no health benefit over less costly alternatives.  According to the Congressional budget office 30 percent of United States spending is wasteful or not necessary \cite{consumer}.  Some waste is the result in the uncertainty in the science of medicine. An example would be when a patient is misdiagnosed or when the treatment protocol is uncertain. Often a newer or more modern treatment is marketed and sold even when it does not provide a better outcome as to compared to the traditional treatment. An example was a 2 million dollar prostate cancer machine that was being marketed in 2001. It made the price of the procedure significantly more, but it did nothing to improve the health outcome \cite{consumer}.  An important factor is that neither the patient nor the provider has any financial incentive to reduce waste or to avoid care that only provides marginal benefits such as overscreenng, excessive follow-up visits, the use of branded instead of prescription drugs \cite{milbank}.  Examples of types of treatment with clinical waste include avoidable emergency room use, unnecessary hospital admissions, and excessive antibiotic use \cite{milbank}. 
\subsection{Medical Errors}
Medical errors cost the United States system between 17 and 29 billion dollars annually \cite{Milman}. This amount could be as much as 1 trillion dollars a year if lost productivity is taken into account \cite{www-google-pracfus}. This is compares to an estimate of 750 million in Canada \cite{Milman}.  The Institute of Medicine estimates that preventable medical errors claim between 44000 and 98000 lives in hospitals each year \cite{milbank}.
\subsection{Fraud and Abuse}
The National Healthcare Anti-Fraud Association estimates losses due to health care fraud at 80 billion dollars annually. Other industry sources estimate fraudulent related losses to be around 200 billion. This accounts for approximately 2 to 3 percent of total health care spending. Research indicates that only 5 percent of these losses are ever recovered \cite{www-google-datameer}.

\section{Big Data}

Big data refers to electronic data sets that are so large and complex that they cannot be managed with traditional hardware and software. A report delivered to the United States Congress in August 2012 defines big data a “large volumes of high velocity, complex, variable data they require advanced techniques and technologies to enable capture, storage, distribution, management, and analysis of the information”. Big data characteristics include variety, velocity, and veracity, and volume.  Health care data is involves the processing of overwhelmingly large complex data sets, from a wide variety of sources and a very rapid speed \cite{springer}.  Health care professionals are now able to capture and analyze these mountains of digitized health care data. Recent advances in Big Data technology gives us the ability to capture, share and store healthcare data at an unprecedented pace. Industry information exists in many formats including images, video, text, numerical, multimedia, paper, and electronic records. The data is extremely difficult to sort, organize, and decipher.  Data and terms may be defined differently from one system to the next \cite{www-google-digit}

\subsection{Volume}
The health care industry has always generated large amounts of data.  Data is needed for record keeping, compliance and regulatory reporting and patient care. Historically this data has been stored in hard copy format. Now more and more data is being created and stored digitally. Estimates are that in 2011 there were 150 Exabyte of data. The amount of big data being generated and is growing rapidly. It is expected to soon reach the zettabyte scale and then soon after that into the yottabytes \cite{springer}.}

\subsection{Velocity}
Traditionally, health care data has been static: paper files, x-ray films, prescriptions. Ironically, in many medical situations, the speed of the response can mean the difference between life and death. Increasingly, more and more of the data is being collected in real time and at a rapid pace. For example medical monitoring devices information is being collected continuously and many time requires immediate response \cite{springer}. 

\subsection{Variety}

There is an enormous variety of data being collected. The data is in multimedia formats. The data may be includes structured, unstructured, or semi-structured. Examples of data include web and social media. These may include Facebook, twitter, health plan websites and smart phone applications. There is machine to machine data from patient sensors. Biometric data is available such as fingerprints, genetics, hand writing information and imagining reports \cite{springer}. Sources of data include physicians, hospitals, laboratories, research companies, insurance companies, and government agencies. Physicians generate electronic medical records, physician notes and medical correspondence. Pharmaceutical companies maintain research and development information in medical databases. Detailed information patient information is available in public insurance and hospital databases. The United States government houses databases concerning clinical drug trials. Data is collected by the United States Centers Disease Control and Prevention \cite{www-google-CIO}. Data is being continuously generated by various medical sensor devices. 

\subsection{Veracity}

The characteristic Veracity addresses whether the information is credible and error free. Veracity is extremely important in health care because life or death decisions on being based upon the information provided. There is a particular concern because interpretations of unstructured data such as physician notes could be incorrect or imprecise. Big data architecture, platforms, methodologies and tools are designed to take into account the uncertainties of big data analytics \cite{springer}. 

\subsection{Infrastructure}
Standards and incentives for digitizing and sharing of healthcare data. 
The Health Insurance Portability and Accountability Act (HIPAA) establishes national standards for electronic healthcare transactions for the submission of claims. Claims are the documents that health provider submit to insurance companies to get paid. Such standards makes document exchange more efficientenables and encourages the widespread use of Electronic Document exchange efficients of effective capability to share and exchange medical information between providers and insurance companies \cite{www-google-McDonald}.
Medicare and Medicaid have set up Electronic Medical Record (EHR) incentive programs to encourage professionals and hospitals to adopt and demonstrate meaningful use of EHRs. 
The costs for storing and parallel processing are decreasing \cite{www-google-McDonald}. 
Much of this information has been underutilized or not being used to its full potential until now. It is estimated that approximately eighty percent of the data is in an unstructured format \cite{www-google-McDonald}. For example, big data gives us the ability to capitalize and make use of the valuable clinical information that is unstructured, such as the information that is available in physician and nurses notes \cite{www-google-HlthCat}.  Advances in Big Data technology including data management and cloud computing are facilitating the development of platforms for more effective capture, storage, and manipulation of large data sets from multiple sources \cite{springer}.  All of this is providing information and insights that has never before been possible to obtain.
Unstructured data makes up about 80 percent of the health care information and is growing exponentially. Getting access to the information that is available in this unstructured output such as medical devices, doctors notes, medical correspondence is an invaluable resource for impoving patient care and increasing efficiency \cite{www-google-McDonald}. In addition, the data 
\section{Value Based Reimbursement}
Outcome based reimbursement model \cite{digit}.
The goal of value based reimbursement structures are to align payment incentives with the administration of efficient high quality medical care. Coupling provider reimbursement with performance and patient outcomes encourages providers work towards optimizing patient health instead of just provider health care services . It also gives caregivers the incentive to more innovative and to search for ways improve health care delivery \cite{www-google-christian}.
Currently, the most common provider payment structure is fee for service. Under this concept, providers are paid a fee for each and every service that they perform. This tends to encourage overutilization instead of the efficient use of medical resources. The United States tends to perform more and more expensive diagnostic services and treatment services than any other country in the in the world. The United States is well known for over testing and over treatment \cite{www-google-PBSO}. The physicians use their own personal judgement to determine what tests and treatment services to order.  From a clinical perspective, many of these tests are not medically necessary. This is a wasteful use of resources. 
For this reason many payers including private health insurance companies, Medicare and Medicaid are starting to base reimbursement of value based incentives.  Value based reimbursement systems reimburse health care providers for health care quality instead of health care quantity. Providers are awarded for high quality and positive patient outcomes instead of service volume \cite{www-google-McDonald}.  Payers are also are starting to reward pharmaceutical companies by basing reimbursements of drug effectiveness \cite{www-google-McKinsey}. 
There are several different types Value based models. One form is the patient centered medical home. In this model the primary care physician coordinates the patients care and is rewarded for performance for improving quality and reducing costs for individual patients. Another value based system is a population model that include population health initiatives and reward providers for successfully improving the health of the entire population \cite{www-google-liason}. An example of this type of program is an Accountable Care Organization (ACO). In Accountable Care Organizations groups of doctors, hospitals, and other providers work together to provide coordinated care for Medicare patients. Providers share in Medicare savings with they deliver high quality care and manage costs wisely \cite{www-google-ACO}.
One example is Memorial Care, which is a six hospital system in Fountain Valley, California. Memorial Care uses physician performance analytics to analyze performance of hospital doctors and outpatient providers.  So far, such tracking has resulted in the reduction 280 dollars per hospital stay for the average adult patient. This equates to a 13.8 million dollar savings for the Fountain Valley Hospital system \cite{www-google-Datafloq}.
The goals of value based care include improved efficiencies, improved care coordination, elimination of unnecessary tests and duplication of services.  All of this results in lower costs, better health outcomes, and enhanced patient satisfaction. 
Big Data Analytics is integral to making this process work. Value based health care depends upon quality data collection and precise data analytics \cite{www-google-liason}.  First, the data must be collected and analyzed in order to define what defines quality care. Big Data is collected and analyzed in order to establishing clinical guidelines that promote a more rational use of specific diagnostic tests and treatment protocols.  Second, this information must be made available to health care givers in a format that they can use for day to day clinical decision making. This is often in the form of a cloud based integration platform \cite{www-google-liason}.  Next, data must be collected on an ongoing basis to provide feedback indicating whether the providers are meeting the defined standards and if not, what can be done to improve performance. In addition, the same data can benefit future patients when data analytics are taken beyond the individual patient and are used to develop care protocols for entire patient populations \cite{www-google-liason}. 

\section{Public Health}
Managing population health by detecting vulnerabilities within patient populations during disease outbreaks or disasters and bringing financial and operational data together to analyze resource utilization productively and in real time \cite{springer}.
Turning large amounts of data into actionable information that can be used to identify needs, provide services, and prevent crises for the benefit of populations \cite{springer}.
Analyzing disease patterns and tracking disease outbreaks and transmission to improve public health surveillance and speed response. Faster development of more accurately targeted vaccines.  
Big data can identify areas in which education and prevention are needed to create healthier populations /cite{www-google-christian}. Healthier populations mean lower costs
\section{Genomic Analytics}
Execute gene sequencing more efficiently and cost effectively and make genomic analysis a part of regular medical care decision process. 

\section{Transparency}
Pay for performance reimbursement agreements encourages health care providers to share information \cite{www-google-christian}.
Reference pricing
\section{Evidence Based Medicine}
Eliminates guesswork for healthcare providers. Instead of having to rely only on their own personal judgement, providers can base treatment and protocols on credible scientific data \cite{www-google-christian}.
Physicians have traditionally relied upon their judgement to make treatment decisions. Now there is a move towards evidence based medicine. Systematically reviewing clinical data and making treatment decisions based upon the best available information.  Aggregating individual data sets into big data algorithms often provides the most robust evidence since nuances in subpopulations may be so rare that they may not be readily apparent in small samples. An example would be presence of patients with gluten allergies \cite{www-google-McKinsey}.
Combine data from both structured and unstructured data from a variety of sources including electronic medical records, financial and operational data, clinical data, and genomic data. Objectives are to match treatments with outcomes, predict patients at risk for a disease or readmission, thereby allowing capability for more efficient and effective care \cite{springer}.
One area in which predicting patient at risk can yield the greatest results in the identifying individuals who are the consumers the most resources and those patients who are at the greatest risk for the most adverse outcomes or costliest diseases \cite{springer}. 
Risk stratification is a methodology that can be used to identify and track the sickest and potentially costliest patients. The tool ranks or stratifies patients by potential by risk and are flagged for additional management. risk stratification predictive tool takes takes into account risk factors such as missed doctors appointments in addition the symptoms. Enables doctors to intervene earlier to avoid hospital admissions and costly treatment \cite{datafloq}. 
When providers have insight on risk for illness populations they can provide proactive responses \cite{www-google-christian}. 
\Drug Costs
Improved Medication Therapy Management - Adverse drug events costs billions of dollars and result in thousands of patient deaths.  Physicians and pharmacist are often overwhelmed to the point having the time to implement appropriate drug therapies. Drug therapies are becoming more difficult to manage as more patients are taking multiple medications.  Big Data cloud analytics helping clinicians better comanage drug therapies and can identify drug interactions, adverse side effects, additive toxicities in real time. Result reduces patient deaths and reduces health care costs fewer emergency room visits, and hospital admissions and readmissions \cite{datafloq}.
\section{Fraud and Abuse}
The National Healthcare Anti-Fraud Association estimates losses due to health care fraud at 80 billion dollars annually. Other industry sources estimate fraudulent related losses to be around 200 billion. This accounts for approximately 2 to 3 percent of total health care spending. Research indicates that only 5 percent of these losses are ever recovered \cite{www-google-datameer}.
Common types of abuse include: billing for services that are not rendered, billing for more expensive procedures than were actually delivered, and the performance of unnecessary services. 
In the past the identification of misrepresented claims was tedious and time consuming work.  According to an article by RevCycle Intelligence, ‘’fraud involves a misrepresentation of some key fact or event and when repeated misrepresentations are made, they create patterns that can be detected when compared to legitimate claims’’ \cite{www-google-datameer}.  Big Data analytics makes it possible to easily identify and tag such claims. Anthem Health Insurance company, one of the nations biggest players uses big data and machine learning algorithms to tag suspicious claims on a daily basis as the claims are processed. Tagged claims are then sent to clinical coding experts for review. The objective is to identify and weed out fraudulent claims before they are actually paid \cite{www-google-datameer}.   
The Center for Medicare and Medicaid Services used predictive data analytics to identify and recover 210.7 million in health care fraud. They did this by assigning risk scores to claims, providers via algorithms, identify abnormal billing patterns and claim submissions.  United Healthcare realized a 2200 percent return on their investment in a Hadoop Big Data platform that was used to identify and tag inaccurate claims using a systemic and repeatable methodology \cite{www-google-McDonald}.
Other uses of Big Data analytics include identifying links between providers to access whether an identified unethical activity is being practiced by related providers.  Big Data analytics can utilize machine learning algorithms combined with historical information to detect anomalies and suspicious data patterns. Big data analytics can be used to identify a hospital’s overutilization of services in a short time period, patients who are receiving health care services from different hospitals in different locations at the same time, or identical prescriptions filled for the same patient in multiple locations. 

\section{Internet Connected Medical Devices}. 
Preventative care applications.  Monitor potential health issues using mobile apps. Fitbit Quickly and more easily communication with physicians. 
Capture and analyze in real time large volumes of fast moving data from in hospital and in home devices for safety monitoring and adverse event prediction /cite{springer}. Real time systems analysis improves patient care and reducing healthcare costs /cite{www-google-christian}.
Improving outcomes by examining vitals from at home health monitors /cite{springer}.
Healthcare facilities can provide more proactive care by constantly monitoring patient vital signs \cite{www-google-McDonald}.
Internet connected medical devices are becoming more affordable and are being used more and more commonly.  Gartner, the analysis firm estimates that there will be more than 25 billion connected health devices by the year 2020 \cite{www-google-HlthCat}. These devices collect data in real time and send information into the cloud. Devices include blood pressure monitors, pulse oximeters, glucose monitors, and electronic scales. \cite{www-google-HlthCat}.  Some of these devices are being used as personal monitors. For example, smartphones can track user exercise.  Patients are able to take a more active role in their own health and wellbeing.
Provide individuals with the information that they need to make informed decisions and more effectively manage their own health and more easily track and adopt healthier behaviors /cite{milbank}.
Other devices being can be worn by patients with health conditions at home and in medical facilities to stream data continuously to provide real time patient monitoring.  Processing or real time events can be supplemented with machine learning algorithms to help provide physicians with information they need to make lifesaving interventions \cite{www-google-McDonald}.  The devices can improve care by enabling patients with the ability to self-manage their conditions. Patient care tends to be more proactive as patient vital signs are can be monitored constantly. Medical alerts can be sent to care providers such that they immediately aware of changes in a patients condition and can respond accordingly. One example of the use of personal devices in patient care are pediatricians monitoring asthmatics to identify environmental triggers for attacks \cite{www-google-CIO}.
\section{Patient Profile Analytics}
Use advanced analytics to patient profiles to identify individuals who would benefit from proactive care or lifestyle changes. Ana example would be patient who are at risk for developing a specific disease such as diabetes who could benefit from preventive care \cite{springer}. 
\section{Electronic Health Records}

Between 2001 and 2014 Electronic Health record usage in physician offices rose from 20 percent to 82 percent. This amount of data that is being collected by large health systems and treatment centers around the country is massive \cite{www-google-pred}. 
Electronic medical records are basically a digitized version of a patients medical chart. Electronic Health records take digitization a step further. Electronic Medical records are digitized versions of what is available in a patients charts from multiple providers. This enables the sharing of information between providers.  The records are made available to health providers in real time. The objective is to provide in one place an electronic record of a patients health. An electronic health record contains medical history, diagnosis, medications, immunizations dates, allergy information, radiology images and test results.  An Electronic Health record system can also automatically check for problems such as medication conflicts and notify clinicians with alerts. Systems often include and electronic prescription subscribing systems. They can include and be integrated with evidence base tools that help providers make decisions about patients care \cite{www-google-elec}.  
In the past the providers had to limit the decisions to the amount of information that was available to them. The planning of care of chronically diseased patient that had many symptoms was often delayed. Makes it easier to treat chronic illnesse and diseases. Electronic Medical Records (EMRS) enable the physicians to facilitate personalized treatment for these patients that has never before been possible. Providers have a comprehensive record of historical treatments, diagnostic data, medical history, and meticulous medical information all in one place. More efficient and effective treatments for chronically ill patients. The result is cutting costs while also reducing potential of side effects and increase patient quality of life \cite{www-google-christian}. Sharing information productivity is increased, overlap is reduced, coordination of care is enhanced \cite{www-google-christian}.
The concept is simple, but it improves health care for the patient in many ways. Physicians have better organized, more accessible, and more complete information about the patient.  Improves ability to make an accurate diagnosis.  Easily accessible patient information reduces medical errors and unnecessary tests. It reduces the incidence of duplicate tests. It improves coordination of care because it insures that all caregivers are aware of the conditions that are being treated by other caregivers. It makes it easier to communicate critical clinical information to all applicable providers in a timely fashion. Because information is made available in real time to providers in drastically reduces the probability of errors caused by such things as allergic reactions and drug interactions especially in emergency situations.  Easier to track and manage when patients are due for vaccinations or screenings. Electronic subscribing allows physicians to communicate directly with pharmacies. Prescriptions are no longer lost or misread \cite{www-google-elec}.
Electronic medical records save money by reducing transcription costs, eliminating chart storage and access costs, reduced medical errors, and improved patient care. At the same time quality of care and patient safety is enhanced, and health outcomes are better \cite{www-google-elec}. 

\section{Predictive Analytics}
As initiatives are encouraging the meaningful use of Electronic Health records the volume and detail of patient information is growing rapidly.  Structured and unstructured data from multiple sources is combined. Improves accuracy of diagnosing patient conditions, matching treatments with outcomes, and predicting specific patients at risk for disease. Enabling Early diagnosis is reducing mortality rates from congestive heart failure. Congestive Heart failure account for more medical health spending than any other diagnosis. The earlier it is diagnosis, the easier it is to treat and the easier it is to avoid expensive complications. However, early manifestation can easily be missed by physicians. Machine learning algorithms have the ability to take into account many more factors than doctors alone. The use of machine learning in the form of predictive analytics there was a substantial increase in the ability to accurately diagnosis persons with congestive heart failure. Predictive modeling and machine learning using large sample sizes can identify nuances and patterns that were previously impossible to see.  An example is Optum labs. They have collected Electronic Medical records to create a database for over 30 million patients. They use the database to develop predictive analytic tools, the objective of which is to help doctors make Big data informed decisions that will improve patients treatment \cite{www-google-McDonald}.  
An example is in Parkland Hospital in Dallas Texas uses predictive modeling to identify high risk patients in the coronary care unit and to predict likely outcomes when the patients are sent home. To date, Parkland has reduced readmissions for Medicare patients with heart failure by 31 percent. This equates to a 500000 dollar annual savings for this one hospital \cite{www-google-datafloq}. 

Predictive analytics is the process of learning from historical data in order to make predictions about the future. The objective of predictive health analytics is to provide insights that enable personalized medical care for each individual patient. Traditionally, physicians have always used predictive analytics as they provide health care based upon what they know about the medical history of each individual. Predictive Health analytics seeks to supplement that knowledge with software tools that enable physicians to make more informed choices about the patients treatment based upon data from population cohorts /cite{pred}.  The patients are directed to specific treatment plans based upon their specific conditions as compared to other patients in a similar cohort.  This additional knowledge can have the potential to provide the physician with the information to provide a more effective treatment plan \cite{www-google-pred}. Predictive analytics can also aid in the diagnosis of a patient. Data sources are disparate and include: clinical, claims, research and patient generated data. Sources of patient generated are sensors and social media.  
Predictive analytics has the potential to materially reduce health care costs and improve patient care.   Insights provided can in clinical decision support, prevent hospital readmission preventions, aid in adverse incidence avoidance, and help chronic disease management. 
Identifying treatments, programs, and processes that do not deliver demonstrable benefits or that cost too much /cite{springer}. Reducing readmissions by identifying environmental of lifestyle factors that increase risk or trigger adverse events and adjusting treatment plans accordingly \cite{springer}.   
An example the Society fro Imaging Informatics in Medicine (SIIM) . Hundreds of thousands of dollars are spent on cancer care. Big data can be used to develop individualized, personalized cancer care programs. There is a web based application that uses data from  the Prostate, Lung, Colorectal, and Ovarian Cancer Screening trial together with patient risk factor and demographic data to help develop patient specific treatment regimes. that was sponsored by the National Cancer Institute,  
 \section{Use Cases}
Valence Health has built a data lake that they use as their primary data repository. using MapR Converged Data Platform. The system includes 3000 inbound data feeds and contains 45 different types of data including:  lab test results, patient vitals, prescriptions, immunizations, pharmacy benefits, claims information from doctors and hospitals. The system reports dramatically better system performance. For example, previously, it took 22 hours to process 20 million laboratory records. Now the processing time for the same number of records is 20 minutes. In addition, the new system requires less hardware \cite{www-google-McDonald}. 
The National Institute of Health developed a data lake which combines data sets from separate institutions. Now that all of the data is housed in the same location, analysis is more efficient and can be more easily shared \cite{www-google-McDonald}.
United Healthcare users an Hadoop to maintain a platform with tools that they use to analyze information generated from claims, prescriptions, provider contracts, and plan subscriber information, and review information \cite{www-google-McDonald}.
Example of this technology is Novartis, a global healthcare company uses Hadoop and Apache Spark to build a workflow system to aid in the integration, processing, and analysis of Next Generation Sequencing research as it relates to Genomic Analytics \cite{www-google-McDonald}.


\section{Challenges}
There are still roadblocks to 
In health care, the privacy, security of the patient is paramount. The Health Insurance Portability and Accountability Act (HIPPA) is a federal law that was passed in 1996 that sets a national standards to protect the confidentiality of medical records and personal health information. The HIPAA law is applicable to any component of the information can be used to identify a person. The protections apply to both electronic and non-electronic forms of information \cite{HIPAA}. HIPAA regulations make it a federal offense to breach patient security.  However, big data technology has inconsistent security technology that makes is potentially problematic. There are some vendors that have experience with security \cite{HlthCat}. Liason Technologies is another company that provides solutions to the healthcare and life sciences industry that meet the HIPAA security requirements \cite{www-google-McDonald}.  
Proliferation of data formats and representations.
Another issue is lack of technical expertise. The manipulation and extraction of data from often unstructured data sets requires special expertise. There have been some recent changes in tooling that will make it easier for individualized with less specialized skills to manipulate the data. For example, Big data is starting to use include SQL as a tool for querying and manipulating the data. Microsoft Polybase and Impala uses SQL Hadoop. 
 

\section{Conclusion}
Is the health care industry poised to take to use big data to its full potential. Innovative technology, significant insights.
Capturing sharing storing vast amounts of healthcare data and transactions, expeditious processing of big data tools have transformed the health care industry by improving patient outcomes while reducing costs \cite{www-google-christian}.
Big data cloud platforms – new era – high quality patient care lowering costs







\begin{acks}

  The author would like to thank Dr. Gregor von Laszewski and the teaching assistants in the Data Science department at Indiana University for their support and suggestions to write this paper.

\end{acks}



\bibliographystyle{ACM-Reference-Format}
\bibliography{report.bib} 

\end{document}
