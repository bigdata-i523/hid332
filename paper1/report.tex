\documentclass[sigconf]{acmart}

\usepackage{graphicx}
\usepackage{hyperref}
\usepackage{todonotes}

\usepackage{endfloat}
\renewcommand{\efloatseparator}{\mbox{}} % no new page between figures

\usepackage{booktabs} % For formal tables

\settopmatter{printacmref=false} % Removes citation information below abstract
\renewcommand\footnotetextcopyrightpermission[1]{} % removes footnote with conference information in first column
\pagestyle{plain} % removes running headers

\newcommand{\TODO}[1]{\todo[inline]{#1}}

\begin{document}
\title{Big Data Analytics in Agriculture}


\author{Judy Phillips}
\orcid{xxxx-xxxx-xxxx}
\affiliation{%
  \institution{Indiana University}
  \streetaddress{PO BOX 4822}
  \city{Bloomington} 
  \state{Indiana} 
  \postcode{47408}
}
\email{judkphil@iu.edu}

% The default list of authors is too long for headers}
\renewcommand{\shortauthors}{B. Trovato et al.}


\begin{abstract}
The modern agricultural industry faces numerous challenges. The global population is increasing rapidly. As the global population grows the agricultural industry must find a ways to increase to the production out of each and every acre in order to match the growing need.  Many parts of the world face food insecurity issues. Food often spoils during food transport as a result of inefficient food packaging or problems with food transportation. This results in substantial food waste and potential health hazards. Big Data and data science is now being adopted into the agricultural industry to help solve some of these issues. The necessary technology is becoming increasing available and is increasing affordable. The use of wireless technology, the Internet of Things, smart devices, and sensors is becoming commonplace throughout the world. Big data is being collected, analyzed, and delivered to back stakeholders to enable better management of operational activities. As a result, farming production processes are becoming more productive and food delivery systems are becoming more reliable.
\end{abstract}

\keywords{I523, HID332, precision farming, smart farming, food production, food safety,  precision agriculture, big data}


\maketitle

\section{INTRODUCTION}

Big Data is revolutionizing the Agricultural Industry. The Internet of things together with the availability of cloud technology is creating a new phenomenon called Smart farming \cite{Wolfert}. Big data is a term for datasets that are so large or complex that traditional data processing applications are not inadequate to process them \cite{Wolfert}. Large amounts of information is being captured, analyzed, and used to make operational decisions \cite{DevEcon}. As a result, farmers are optimizing productivity, reducing costs, reserving resources, and increasing profitability. 
 

Big Data Analytics is also reducing waste and spoilage as food moves through the food supply chain.  According to McKinsey and Company, approximately one-third of all food in lost or wasted every year.  That equates to a nine hundred forty (940) billion dollar Global impact \cite{www-google-bigdatatech}.  Much of this occurs during the food shipment process.  

Internet connected devices are becoming common place on farms. Almost all new farm equipment has sensors. Sixty percent of farmers report some type of internet sourced data to make operational decisions \cite{www-google-Farm}. Sensors are becoming common in food packaging.  The related software market is growing rapidly.  In 2010 the investment in Agricultural Technology was 500 million. In 2015 the investment had grown to 4.2 billion \cite{DevEcon}. 

\section{BIG DATA}

Big data represents information assets characterized by such high volume, velocity and variety as to require a specific set of technology and analytical methods for its transformation \cite{Wolfert}. The amount of data and information generated by the food production industry is massive. For example, it is estimated that sensors on harvesting equipment generate about seven gigabytes per acre. There are 93 million acres of corn and 80 acres of soybeans in the United State alone \cite{www-google-Farm}. In India, there are one billion acres. Data is being collected at the micro-bit level and much of this data is being processed in real time \cite{www-google-Crop}.  

\section{THE SMART FARM AND PRECISION AGRICULTURE}

\subsection{Precision Agriculture - Overview}

Precision agriculture is a specific farm management technique that uses sensor and analytic technology to measure, observe and respond to crop and livestock management in real time.  Precision farming matches farming techniques to the specific crop and livestock needs.  The objective of precision farming is to ensure that crops receive that exact inputs that they need, at the correct time, and in precise amounts \cite{www-google-Wikipedia}. Examples of crop inputs include: water, fertilizer, herbicides and pesticides.  This strategy enables a farmer to get the most productivity out each and every resource. Solutions are customized to each individual farmers unique needs.

Processes that are typically managed with Precision techniques include: seeding, planting, harvesting, weed control \cite{www-google-Digital}, fertilizer management, breeding, disease control, pesticide management, light and energy management \cite{Wolfert}. 

\subsection {Precision Farming - Benefits}

Precision farming techniques give farmers the ability to make operating decisions in real time based upon data and information that is being generated in real time.  It also gives farmers the ability to make predictive insights in farming operations \cite{Wolfert}.  All of this results in significant benefits: Increased yields, reduced costs, greater productivity, immediate disease management \cite{www-google-Crop}, improved crop quality, and better cash flow.  Big Data makes farms more profitable. Also, when inputs such as herbicides and pesticides are better managed, it helps the environment. Precision farming also has a socioeconomic impact worldwide because efficiency improvements can help to alleviate global food insecurity \cite{Wolfert}.

\subsection{Precision Farming - Data Collection}

A very common approach to collecting data is sensor technology.  Sensor technologies measure and monitor data. Sensors register and report deviations in real time.  Sensors include devices that are located locally on the farm and external satellites. 

Types of local sensors include: connected farming equipment (tractors, harvesters), chips planted into livestock \cite{DevEcon}, and drones.  Examples of the types of data that may be collected via local sensors include: Rainfall and water measurements, crop health, livestock health, weather information, yield monitoring, and lighting and energy management \cite{Wolfert}.  Drones can collect aerial images of fields.  Aerial field images can help to monitor crop health. \cite{www-google-Wikipedia}. Data is oftentimes collected in very precise detail.  For example, information can be gathered for for each square meter of land or for every individual plant \cite{www-google-Digital}. 

Data collected with local sensors is often supplemented with information other external sources such as satellites and the cloud.  Data that may be collected via satellite and available in real time on the cloud includes: Weather and climate data (historical and real time), soil type analysis, market information, and livestock movements.  Data collected from orbiting satellites can also be very granular and personalized \cite{DevEcon}. For example, soil characteristics such as texture, organic matter, and fertility is collected to the meter at locations throughout the world \cite{DevEcon}. 

\subsection{Precision Farming - Data Analysis}

After the data is collected it must me consolidated and analyzed.  A significant amount of this support is being provided by machine supplier companies that have been servicing the farming industry for generations such as John Deere, DuPont Pioneer, and Monsanto \cite{www-google-Crop}.  Now, in addition to selling seeds and machinery, these companies are selling decision support and data science services \cite{www-google-Farm}.

Most of this support is in the form of software decision support technology.  Companies collect information from individual farms, combine this information with data other sources, including their own databases, and apply statistical  models and algorithms.  Results and recommendations are delivered to each grower as personalized solutions.  Examples of some potential solutions are: how far apart to place seeds based upon the field position, or what to do to better manage nitrogen levels in the soil \cite{www-google-Crop}. 

These companies have developed and maintain massive databases of their own.  DuPont Pioneer has mapped and has collected data on 20 million acres in the United States. Another company, Cropin, which provides support for farmers worldwide, including growers in extremely remote areas, has mapped over one billion acres globally. Cropin can provide data by individual farm, farm clusters, districts, states, and even countries (India) \cite {www-google-Crop}.

In addition to big companies, there are also public institutions that are involved with Big Data Applications. These include universities, the USDA, and the American Farm Bureau Federation.  Their interest typically involves issues such as food safety, food security, and data privacy regulation \cite{Wolfert}. 

\subsection{Precision Farming - Infrastructure}

After the data is analyzed it is downloaded from the cloud and made available to the farmers, typically through wireless technology devices. It may be downloaded to an farmers Ipad or computer in a tractor.  Other information can be sent to Smart phones. By interacting with the Internet of Things farmers can manage operational activities from anywhere in the world \cite{Wolfert}.  Other devices are self automated. One such self automated technology is Variable rate technology (VRT). Variable rate technology is built into equipment such as irrigation systems, feeders, and milking devices \cite{www-google-Wikipedia}. These devices automatically operate in such a way as to deliver optimal results with no human intervention. 

None of these processes can happen without the appropriate infrastructure to store, transmit, and transform the data. Typical Storage vehicles for this data are typically cloud based platforms, Hadhoop Distributed file system, cloud based data warehouses and hybrid systems.  Data transfer is accomplished via wireless technology using cloud based platforms. Machine learning algorithms are typically used to transform and cleanse the data \cite{Wolfert}.

\subsection {Precision Farming - Decision Making}

Below are some examples of ways in which information provided by Big Data Analytics is providing farmers with the information that they need to make more informed decisions concerning their operations.

Following are some examples of technology in the world of crop science: Satellite systems and sensors can monitor the development of crops in detail. Individual plants can be monitored for nutrients, growth rate and health \cite{www-google-bigdatatech}. In this way,  disease outbreaks can be recognized and addressed immediately \cite{www-google-Crop}.  Entire fields can be mapped with GPS coordinates to collect data concerning soil conditions and elevation. The data in analyzed using Algorithms and the data is sent back to an Ipad on the farmers tractor. The tablet then communicates with the tractor's planting mechanism telling it exactly where to place every seed \cite{www-google-Crop}. This same technology can even tell if a single seed has been missed \cite{www-google-Farm}. GPS units on tractors, combines, and trucks help determine the optimal usage of equipment \cite{www-google-bigdatatech}.

Big Data technology also improves the field of Animal and livestock management. Milk cows are tagged with chips that monitor the health of the animal. Milking machines shut down when the animal is sick \cite{DevEcon}.  Sensors indicate when livestock are ready to inseminate or give birth \cite{www-google-Digital}. Smart dairy farms are using robots to complete tasks such as feeding cows, cleaning barns, and milking cows \cite{Wolfert}. 

Consolidated data can offer insights and information that has never before been possible. Big data companies can test and gather information about the effectiveness of different kinds of seeds across many different conditions, soil types, and climates. The origin of crop diseases can be identified quickly and efficiently with web searches similar to the way that flu epidemics are currently identified \cite{www-google-Farm}. This will enable players to take corrective action quickly. Historical analytics can determine the best crops to plant \cite{Wolfert}. 


\section{Food Safety and the Food Supply}

Big Data not only impacts primary food production, it helps to improve the entire food supply chain \cite{Wolfert}. According to the Food and Drug Administration,food waste equates to approximately 680 billion in industrial countries and 310 billion in developing countries annually \cite{DevEcon}.  A significant amount of this food waste occurs during food transport. Big Data can help to address this issue in various ways. First, it can help to manage the logistics of transportation. For example, Big Data can help to insure that food is transported in the best weather conditions in developing countries.  This helps to avoid issues such as trucks not be able to navigate muddy roads. Big data can also assist coordination needs between supplier, retailer, and consumer. For example, consumer demand can be tracked with customer loyalty cards or retailers data on shopping patterns.  Coordinating food delivery with consumer need helps to minimize food waste \cite{DevEcon}.

Food spoilage can also be monitored during food transport. Inadequate packaging of food often results in food waste and food spoilage that can even result in life threatening food borne illnesses \cite{www-google-bigdatatech}.  Packaging sensors can detect gases that is being emitted from food when it starts to spoil.  RFID based traceability systems can monitor food as it moves through the supply system. Packaging integrity and freshness can be monitored in real time. Therefore, waste is reduced and food quality issues can be addressed as they occur \cite{www-google-bigdatatech}. 


\section{CHALLENGES AND ISSUES}

\subsection{Developing Countries}

The challenges in developing countries are unique.  In order for Big Data to be successful there must be infrastructure.  Technologies such as satellite imagery and weather monitoring may not be fully developed.  Small farmers can not always afford specialized machinery.  Farmers do not always have access to devices such as computers, tablets, or Ipads \cite{DevEcon}. 

Such issues are starting to be addressed in some countries. For example, in Africa organizations are being formed which pool several farmers resources together. This  enables better access to resources as well as educational information.  Also, there are establish companies that are starting to invest and develop technologies around the world, such as CropIn and Monsanto.\cite {DevEcon}.  Mobile devices such as Smart phones are becoming more common and are starting to be used more widely to manage information. For example, in Tanzania 30000 farmers use mobile phones for business purposes such as contracts, loans and payments.\cite{www-google-Wikipedia}.

\subsection {United States}

In the United States, machine suppliers in the form of big companies have played a big role in this evolution by developing decision support tools that provide information to better manage farms \cite{www-google-Crop}.  When individual farmers share their personal data with big companies such as John Deere and Monsanto it raises some significant unanswered questions and concerns. Is my personal data safe? Is my data secure? Who owns the data? Who will profit from the data? \cite{www-google-Farm}.  Even if  it is assumed that the original data belongs to the individual farmers, there is still the question of who owns the data after it is consolidated. Furthermore, there is concern that the aggregated data could be used to for malicious intent such as manipulation of commodity markets \cite{Wolfert}. 

For these reasons, there need to be clear and defined standards regarding issues of privacy, security, data ownership, and market speculation. Such standards are only in the beginning stages of development. Organizations who are currently working on the farmers behalf to develop these standards include: The American Farm Bureau Association, The Big Data Coalition and AgGateway. In the interim, farmers need to do their best to fully understand any contracts that they sign in which they agree to share data.   \cite{Wolfert}.

\section{CONCLUSION}
Big data analytics are improving our food delivery system in ways that are beyond substantial.  Information is being made available to stakeholders that has previously been impossible to obtain. Big data is being referred to as the most significant revolution in farming productivity since mechanization. Today, billions of people worldwide cope with undernourishment and alarming food shortages. Big Data is expected to make an impact food insecurity throughout the world as more farmers adopt these techniques. This technology will enable even small holder farmers to make full use of their productive potential.  Big Data technology is making the food delivery system healthier, safer, and more efficient.  Big Data in agriculture is here to stay.
\begin{acks}

  The author would like to thank Dr. Gregor von Laszewski and the teaching assistants in the Data Science department at Indiana  University for their support and suggestions to write this paper.

\end{acks}



\bibliographystyle{ACM-Reference-Format}
\bibliography{report.bib} 

\end{document}
