\documentclass[sigconf]{acmart}

\usepackage{hyperref}

\usepackage{endfloat}
\renewcommand{\efloatseparator}{\mbox{}} % no new page between figures

\usepackage{booktabs} % For formal tables

\settopmatter{printacmref=false} % Removes citation information below abstract
\renewcommand\footnotetextcopyrightpermission[1]{} % removes footnote with conference information in first column
\pagestyle{plain} % removes running headers

\begin{document}
\title{Big Data Analytics in Agriculture}


\author{Judy Phillips}
\orcid{xxxx-xxxx-xxxx}
\affiliation{%
  \institution{Indiana University}
  \streetaddress{3209 E 10th St}
  \city{Bloomington} 
  \state{Indiana} 
  \postcode{47408}
}
\email{judkphil@iu.edu}

% The default list of authors is too long for headers}
\renewcommand{\shortauthors}{B. Trovato et al.}


\begin{abstract}
This paper discusses ways that Big Data and Data Science is impacting the industry of agriculture and food safety in the food supply chain.
\end{abstract}

\keywords{Precision farming, Smart farming }


\maketitle

\section{Introduction}

Big Data is revolutionizing the Agricultural Industry. The Internet of things together with the availability of cloud technology is creating a new phenomenon called Smart farming. Large amounts of information is being captured, analyzed, and used to make operational decisions.  \cite{book}. As a result, farmers are optimizing productivity, reducing costs, reserving resources, and increasing profitability. 

Big Data Analytics is also reducing waste and spoilage as food moves through the food supply chain.  According to McKinsey and Company,  approximately one-third of all food in lost or wasted every year. That equates to a nine hundred forty (940) billion dollar Global impact. The \cite{www-google}. Much of this occurs during the food shipment process.  

Internet connected devices are becoming common place on farms. Almost all new farm equipment has sensors. Sixty percent of farmers report some type of internet sourced data to make operational decisions. Sensors are becoming common in food packaging. The related software market is growing rapidly. In 2010 the investment in Agricultural Technology was 500 million. In 2015 the investment had grown to 4.2 billion The \cite{book}. 


\section{The Smart Farm and Precision Agriculture}

\subsection{Precision Agriculture - Overview}

Precision agriculture is a specific farm management technique that uses sensor and analytic technology to measure, observe and respond to crop and livestock management in real time. Precision farming matches farming techniques to the specific crop and livestock needs. The objective of precision farming is to ensure that crops receive that exact inputs that need, at the correct time, and in precise amounts. Examples of crop inputs include: water, fertilizer, herbicides and pesticides. This strategy enables a farmer to get the most productivity out each and every resource. Solutions are customized to each individual farmer’s unique needs.

Processes that are typically managed with Precision techniques include: seeding, planting, harvesting, weed control, fertilizer management, breeding, disease control, pesticide management, light and energy management. 

\subsection {Precision Farming - Benefits}

Precision farming techniques give farmers the ability to make operating decisions in real time based upon data and information that is being generated in real time. It also gives farmers the ability to make predictive insights in farming operations. All of this results in significant benefits: Increased yields, reduced costs, greater productivity, better disease management improved crop quality, and better cash flow. Big Data makes farms more profitable. Also, when inputs such as herbicides and pesticides are better managed, it helps the environment. Precision farming also has a socioeconomic impact worldwide. Efficiency improvements help alleviate global food insecurity.

\subsection{Precision Farming - Data Collection}

A very common approach to collecting data is Sensor technology. Sensor technologies measure and monitor data.  Sensors register and report deviations in real time. Sensors include devices that are located locally on the farm and external satellites. 

Types of local sensors include: connected farming equipment (tractors, harvesters), chips planted into livestock, and drones. Examples of the types of data that may be collected via local sensors include: Rainfall and water measurements, crop health, livestock health, weather information, yield monitoring, and lighting and energy management. Drones can collect aerial images of fields. Data is oftentimes collected in very precise detail. For example, information can be gathered for for each square meter of land or for every individual plant. 

Data collected with local sensors is often supplemented with information other external sources such as satellites and the cloud. Data that may be collected via satellite and available in real time on the cloud includes: Weather and climate data (historical and real time), soil type analysis, market information, and livestock movements. Data from collected from orbiting satellites can also be very granular and personalized. For example soil characteristics such as texture, organic matter, and fertility is collected to the meter at locations throughout the world. 

\subsection{Precision Farming - Data Analysis}

After the data is collected it must me consolidated and analyzed. A significant amount of this support is being provided by machine supplier companies that have been servicing the farming industry for generations such as John Deere, Dupont Pioneer, and Monsanto. Now, in addition to selling seeds and machinery, these companies are selling decision support and data science services \cite{Crop}.

Most of this support is in the form of software decision support technology.  Companies collected information from individual farms, combine this information with data other sources, including their own databases, and apply statistical  models and algorithms. Results and recommendation are delivered to each grower as a personalized solutions. Examples of some potential solutions are: how far apart to place seeds based upon the field position, or what to do to better manage nitrogen levels in the soil. These companies have development massive databases of their own.  Dupont Pioneer has mapped and has collected data on 20 million acres in the United States. Another company, Cropin, which provides support for farmers worldwide, including growers in extremely remote areas, has mapped over one billion acres globally. Cropin can provide data by individual farm, farm clusters, districts, states, and even countries (India) \cite {Crop}.

In addition to big companies, there are also public institutions that are involved with Big Data Appliations. These include universitys, the USDA, and the American Farm Bureau Federation. Their interest typically involves issues such as food safety, food security, and data privacy regulation \cite {article}. 

\subsection{Precision Farming - Infrastructure}

After the data is analyzed it is downloaded from the cloud and made available to the farmers, typically through wireless technology devices. It may be downloaded to an farmers Ipad or computer in a tractor. Other information can be sent to Smart phones. By interacting with the Internet of Things farmers can manage operational activities from anywhere in the world. Other devices are self automated. One such self automated technology is Variable rate technology (VRT). Variable rate technology is built into equipment such as irrigation systems, feeders, and milking devices. These devices automatically operate in such a way as to deliver optimal results with no human intervention.

None of these processes can happen without the appropriate infrastructure to store, transmit, and transform the data. Typical Storage vehicles for this data are typically cloud based platforms, Hadhoop Distributed file system, cloud based data warehouses and hybrid systems. Data transfer is accomplished via wireless technology using cloud based platforms. Machine learning algorithms are typically used to transform and cleanse the data. \cite {article}

\subsection {Precision Farming - Decisions Examples}

In this section I will share some examples of how the information provided by Big Data Analytics is enhancing Agricultural decision making and productivity.

Following are some examples of technology in the world of crop science: Satellite systems and sensors can monitor the development of crops in detail. Individual plants can be monitored for nutrients, growth rate and health. In this way disease outbreaks can be recognized and addressed immediately.  Entire fields can be mapped with GPS coordinates to collect data concerning soil conditions and elevation. Algorithyms instruct the tractor's planting mechansim where to place every seed. This same technology can even tell if a single seed has been missed. GPS units on tractors, combines, and trucks help determine the optimal usage of equipment.

Big Data technology also improves the field of Animal and livestock management. Milk cows are tagged with chips that monitor the health of the animal. Milking machines shut down when the animal is sick. Sensors indicate when livestock are ready to inseminate or give birth. 

Consolidated data can offer insights and information that has never before been possible. Big data companies can test and gather information about the effectiveness of different kinds of seeds across many different conditions, soil types, and climates. The origin of crop diseases can be identified quickly and efficiently with web searches similar to the way that flu epidemics are currently identified. This will enable players to take corrective action quickly. Historical analytics can determine the best crops to plant. 


\section{Food Safety and the Food Supply}

Big Data not only impacts primary food production, it helps to improve the entire food supply chain. According to the Food and Drug Administration,food waste equates to approximately 680 billion in industrial countries and 310 billion in developing countrie annually /cite {book}. A signicant amount of this food waste occurs during food transport. Big Data can help to address this issure in various ways. First, it can help to manage the logistics of transportation. For example, Big Data can help to insure that food is transported in the best weather conditions in developing countries. This helps to avoid issues such as trucks not be able to navigate muddy roads. Big data can also assist coordination needs between supplier, retailer, and consumer. For example, consumer demand can be tracked with customer loyalty cards or retailers data on shopping patterns. Coordinating food delivery with consumer need helps to minimize food waste.

Food spoilage can also be monitored during food transport. Inadequate packaging of food often results in food waste and food spoilage that can even result in life threatening food borne illnesses. Packaging sensors can detect gases that is being emitted from food when it starts to spoil. RFID based tracebility systems can monitor food as it moves through the supply system. Packaging intergrity and freshness can be monitored in real time. Waste is reduced and food quality issues can be addressed in real time. \cite{www-google1} 


\section{Challenges and Issues}

\subsection{Developing Countries}

The challenges in developing countries are unique. In order for Big Data to be successful there must be infrastructure.  Technologies such as satellite imagery and weather monitoring may not be fully developed. Small farmers can not afford specialized machinery. Small farmers do not always have access to devices such as computers, tablets, or Ipads.

Such issues are starting to be addressed in some countries, such as Africa with farmers organizations which pool several farmers resources. The enables better access to information and educational resources.  Also companies that are already important players in the related technology in the developing world, such as Mosanto are now starting to develop resources in other places around the world. \cite {book}.

\subsection (United States}

In the United States, machine suppliers in the form of big companies have played a big role in this evolution by developing decision support tools that provide information to better manage farms. When farmers share their data with big companies such as John Deere, there are several concerns including privacy and security and data ownership. Even when it is assumed that the data originally belongs to the individual farmers from which it was sourced, there is still the question of who owns the data after it is consolidated. Furthermore, there is concern that the aggregated data could be used to manipulate commodity markets. There need to be defined standards that address data ownership, data security and market speculation. Organizations that are currently working to define such standards in include: The American Farm Bureau Association,The Big Data Coalition and AgGateway.

\section{Conclusion}

Improvements to agricultural productivity as result of big data technology are beyond substantial. Big data is being referred to as the most major revolution productivity in farming since mechanization.  In 2009, the United Nations estimated that 900 people in the world were undernourished and that 65 countries face alarming food shortages. Big Data is expected to make an impact on Food Insecurity throughout the world as farmers throughout the world adopt these techniques. Will enable even small holder farmers to make full use of their productive potential. The use of precision farming techniques and digital technologies will enable farmers to maximize the use of every inch of soil and even the production of each individual plant.

Big Data is improving the food delivery system. Information is available to producers and suppliers that in the past has been impossible to obtain. Food producers are more efficient and the improvements to the food supply chain has made the food healthier. Big Data is helping to resolve the global issues of food insecurity and food safety in a big way. Big Data in Agriculture is here to stay. 

 \cite{vanGundy09}
The \textit{misc} are the \cite{www-google}

The \textit{misc} are the \cite{www-google1}

The \textit{misc} are the \cite{www-google2}

The \textit{book} are the \cite{book}
The \textit{article} are the \cite{article1}
\begin{acks}

  The author would like to thank the TAs at Indiana University \textit{BEPS} method.

\end{acks}

\bibliographystyle{ACM-Reference-Format}
\bibliography{report.bib} 

\end{document}
\documentclass{article}
\usepackage[utf8]{inputenc}

\title{paper1}
\author{judkphil }
\date{September 2017}

\begin{document}

\maketitle

\section{Introduction}

\end{document}

\documentclass[sigconf]{acmart}

\usepackage{hyperref}

\usepackage{endfloat}
\renewcommand{\efloatseparator}{\mbox{}} % no new page between figures

\usepackage{booktabs} % For formal tables

\settopmatter{printacmref=false} % Removes citation information below abstract
\renewcommand\footnotetextcopyrightpermission[1]{} % removes footnote with conference information in first column
\pagestyle{plain} % removes running headers

\begin{document}
\title{Big Data Analytics in Agriculture}


\author{Judy Phillips}
\orcid{xxxx-xxxx-xxxx}
\affiliation{%
  \institution{Indiana University Bloomington}
  \streetaddress{3209 E 10th St}
  \city{Bloomington} 
  \state{Indiana} 
  \postcode{47408}
}
\email{judkphil@iu.edu}

% The default list of authors is too long for headers}
\renewcommand{\shortauthors}{B. Trovato et al.}


\begin{abstract}
This paper discusses ways that Big Data analytics and Data Science is improving the industry of Agriculture.
\end{abstract}

\keywords{Precision farming, Smart farming }


\maketitle

\section{Introduction}

Big Data Analytics is revolutionizing the Agricultural Industry. Farmers are using the information that is being made available with big data and the Internet things to increase efficiency, optimize results, and reserve resources. Internet connected devices are becoming common place on farms.  Almost all new farm equipment has sensors. Sixty percent of farmers report some type of internet sourced data to make operational decisions. The related software market is growing rapidly. In 2010 the investment in Agricultural Technology was $500 million. In 2015 the investment had grown to $4.2 billion.  In this paper will discuss ways in how Big Data is impacting the field of Agriculture with Smart or Precision farming.

The \textit{proceedings} are the \cite{VanGundy09}

\section{THE SMART FARM AND PRECISION AGRICULTURE}

Precision agriculture is a specific farm management technique that uses sensor and analytic technology to measure, observe and respond to crop and livestock management in real time.
As the name implies, Precision farming matches farming techniques to the specific crop and livestock needs. The objective of precision farming is to ensure that crops receive that exact inputs that need, at the correct time, and in precise amounts. Examples of crop inputs include: water, fertilizer, herbicides and pesticides. This strategy enables a farmer to get the most productivity out each and every resource. Solutions are customized to each individual farmer’s unique needs.

Processes that are managed Precision techniques include: seeding, planting, harvesting, weed control, fertilizer management, breeding, disease control, pesticide management, light and energy management. Benefits of precision farming include improved yields, improved crop quality, reduced costs, and increased profits.  Because fertilizers, pesticides and weed control applications are not being overused, precision farming also helps the environment. 


The lifecycle of the smart farming process is as follows: Sensoring and monitoring, analysis and decision making, and intervention.  We will discuss each of this aspects in more detail below.

Sensor technologies measure and monitor data.  Sensors register and report deviations in real time. Sensors include devices that are located locally on the farm and external satellites. 

Types of local sensors include: connected farming equipment (tractors, harvesters), chips planted into livestock, and drones. Examples of the types of data that may be collected via local sensors include: Rainfall and water measurements, crop health, livestock health, weather information, yield monitoring, and lighting and energy management. Data is oftentimes collected in very precise detail. Data can be to the square meter of land and even in at the individual plant level. 

Data collected with local sensors is often supplemented with information other external sources such as satellites and the cloud. Data that may be collected via satellite and available in real time on the cloud includes: Weather and climate data (historical and real time), soil type analysis, market information, and livestock movements. Data from collected from orbiting satellites can also be very granular and personalized. For example soil characteristics such as texture, organic matter, and fertility is collected to the meter at locations throughout the world. 

\section{Challenges and Issues}

Machine suppliers in the form of big companies have played a big role in this evolution by developing desision support tools that provide information to better manage farms. When farmers share their data with big companies such as John Deere, there is some concern that these big players have the potential to consoldiate the data and use it to manipulate the market. There need to be defined standards for the use of such data. The American Farm Bureau association is seeking data usage assurance regulations within the industry.

\section{Conclusions -Potential Worldwide Impact}

Improvements to agricultural productivity as result of big data technology are beyond substantial. Big data is being referred to as the most major revolution productivity in farming since mechanization.  In 2009, the United Nations estimated that 900 people in the world were undernourished and that 65 countries face alarming food shortages. Big Data is expected to make an impact on Food Insecurity throughout the world as farmers throughout the world adopt these techniques. Will enable even small holder farmers to make full use of their productive potential. The use of precision farming techniques and digital technologies will enable farmers to maximize the use of every inch of soil and even the production of each individual plant.


\begin{acks}

  The authors would like to thank Dr. Yuhua Li for providing the
  matlab code of the \textit{BEPS} method.

\end{acks}

\bibliographystyle{ACM-Reference-Format}
\bibliography{report} 

\end{document}

*********************************************************************************************************************
Notes and Drafts - Do not use:
\documentclass[sigconf]{article}
\usepackage{hyperref}

\usepackage{endfloat}
\renewcommand{\efloatseparator}{\mbox{}} % no new page between figures

\usepackage{booktabs} % For formal tables

\settopmatter{printacmref=false} % Removes Citation information below abstract

\renewcommand\footnotetextcopyrightpermission[1]{} % removes footnote with conference information in first column

\pagestyle{plain} % removes running headers

\begin{document} 
\title{Big Data Analytics in Agriculture}

\author{Judy Phillips }
\affilition{%
  \institution{Indiana University Bloomington}
  \streetaddress{Smith Reasearch Center}
  \city{Bloomington}
  \state{IN}
  \postcode{47408}
  \country{USA}

}
\email(judkphil@iu.edu}


\begin{abstract}
This paper discusses how big data technologies are changing and improving the field of agriculture    
    
\end{abstract}

\keywords (Big Data, Precision farming}

\begin{document}

\maketitle

\section{Introduction}
Big Data Analytics is starting to make significant impacts to the Agricultural Industry. Farmers are utilize data from data various sources to manage operational decisions. Internet connected devices are becoming common place on farms.  Almost all new farm equipment has sensors. Sixty percent of farmers report some type of internet sourced data to make operational decisions. The related software market is growing rapidly. In 2010 the investment in Agricultural Technology was $500 million. In 2015 the investment had grown to $4.2 billion.  In this paper will will discuss ways in which Big Data and Data science is improving the field of Agriculture with precision farming.

\section{Global food insecurity}

\section{History}
To date, there have been two significant stages of agricultural revolution. The first phase, which lasted through the 1920's was the pre-industrial phase. Farming at this time was very labor intensive. One acre could feed 1/2 of a person. The next phase was earmarked with the implementation mechaninzation (tractors, harvesters), plant science, and chemical fertilizers and pesticides.  This phase lasted between 1920 until 2010.  Production had improved significantly - One acre was feeding 5 people. Beginning around 2011 we started to enter the third phase the third phase with the implementation of Big Data Analytics. This phase is the era of Big data analytics and the Internet of Things. Data is now being aggregated from diverse technological sources including  locally placed , and satellites. This information is being analyzed in the cloud back smart devices where it can be used to manage farming operational activitiesin real time. 

It is expected that the digital revolution and Big Data Analytics will play a significant role in meeting the growing food production needs of the future. It is estimated that the world population will be 9.6 billion people in 2050. Food production must effectively double fom what we currently produce in order to feed everyone. The use of precision farming techniques and digital technologies will enable farmers to maximize the use of every inch of soil and even the production of each individual plant

\section{PRECISION FARMING}

Preciesion agriculture is a specific farm management technique that uses sensor and analytic technology to measure, observe and respond to crop and live stock management in real time.
As the name implies, Precision farming matches farming techniques to the specific crop and livestock needs. the objective of precision farming is to ensure that crops receive that exact inputs that need, at the correct time, and in precise amounts.Examples of crop inputs include: water, fertilizer, herbicides and pesticides.

Benefits of precision farming include improved yields, improved crop quality, reduced costs, and increased profits.

Precision farming also helps the environment. Because fertilizers, pesticides and weed control applications are not being overused, precision farming also helps the enviorment. 




Vast amounts of data information is collected from smart machines in the field, satellites and other cloud resources. The data is crunched and analyzed by smart devices and turned into knowledge that can be used to make more efficient operational farming decisions. 


Sensor technologies measure and montitor data. Examples of sensors include: connected farming equipment (tractors, harvesters), drones, and chips planted into livestock. Sensors register and report deviations in real time. External data from the cloud, such as weather information is used to complement the information. 

Examples of the types of data that is collected via local sensors include: Rainfall and water measurements, crop health, livestock health, weather information, yield monitoring, lighting and energy management. Data can be collected at the individual plant level. Data that may be collected via sattelite and available in real time on the cloud includes: Weather and climate data (historical and real time), soil type analyis. market information, livestock movements. This data is at the detail of each square inch of land or each individual plant.

Data is used to manage processes such as: seeding, planting, harvesting, weed control, fertilzer management, breeding, disease control, pesticide management, light and energy management. 

Internet of things
/section{Examples}

* Sensors on fields and crops at granular data points collect data on local soil conditions, wind fertilzer requirement, water availability, and pest control
*GPS Units of tractors, combines, and trucks help determine optimal usage of equipment.
*Individual plants monitored for nutrients and growth rates
*Historical analysis can determine best crops to plant
*Entire fields mapped with GPS coordinates monitoring soil and elevation. Algorithyms tell tractos planting mechanism where to plant every seed. Can even identify if a seed has been missed. 
*Sensors indicate when livestock are ready to give birth or inseminate. 

/section{challenges}

Machine suppliers in the form of big companies have played a big role in this evolution by developing desision support tools that provide information to better manage farms. When farmers share their data with big companies such as John Deere, there is some concern that these big players have the potential to consoldiate the data and use it to manipulate the market. There need to be defined standards for the use of such data. The American Farm Bureau association is seeking data usage assurance regulations within the industry.

section/{conclusion}

Helps farmers - Increases efficiency, optimizes results, reserves resources,
improves profitablity.

Protects enviroment

Global food security for world






\bibliographystyle{plain}
\bibliography{references}
\end{document}
